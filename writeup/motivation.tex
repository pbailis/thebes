
\section{The Case for HATS}

Why High Availability?
* Highly available system operation is often desirable; if you can contact one server, you can get a response!
* As a corrollary, an "AP" system is faster than a "CP" system; see "PACELC" by Abadi.
	* Look, here are some datapoints we recently measured on EC2. Network times are at least 10x faster, often much more for "AP"


Availability examples

Within DC at Microsoft: 40.8 mean link failures per day, 136 at 95th
percentile, with high variability, median time to repair around 5
minutes but tail up to one week. Probability of device failing and
impacting network (per year) is near 10\% for several core components
like core routers and primary-backup connectors. Network redundancy
only reduces median impact of failures by up to
40\%~\cite{sigcomm-dc}. 

HP Labs: WAN, LAN, and connectivity problems account for 28.1\% of all
customer support tickets, 39\% is network hardware, only 6\% for
configuration and 15\% for software. Median duration of highest
priority is on the order of several hours (1:54 for WAN, 2:45 for LAN,
2:13 for Hardware), going to a full day for all
incidents.~\cite{turner2012failure}.

Network partitions happen at Google~\cite{dean-keynote}.

Many anecdotes online: Amazon's EC2 and RDS down due to net
partition~\cite{amazon-netpartition}, as a result of AZ failures full
site outages like Reddit, Foursquare, and Heroku~\cite{ec2-downsites},
accounts of network partitions~\cite{netpartition-anecdote1}

Despite speculation over partition
frequency~\cite{stonebraker2010errors}, many practitioners indicate
that partitions do indeed occur: in the words of James Hamilton, Vice
President and Distinguished Engineer on the Amazon Web Services team,
``network partitions should be rare but net gear continues to cause
more issues than it should''~\cite{hamilton-partitions}

Many of our discussions with practitioners---especially those
operating on public cloud infrastructure---confirm that partitions are
a reality.

Area of active research in networking
community~\cite{surviving-failures-bodik}


Over WAN: 

From 2002 study of Sprint's networks, median time to repair is between 2 and < 1000 seconds, approximately 2000 seconds between router failures and around 8,000 seconds between optical failures. Median failure rate is one every 3000 seconds.~\cite{ip-backbone-failures}.

From 2002 study of VoIP, 50 minute failure caused 100\% packet loss between east and west coasts~\cite{voip-partitions}.

* As an example of a real system going from "CP" to "AP", [PNUTS dropped per-record mastering](\url{http://developer.yahoo.com/blogs/ydn/posts/2010/06/sherpa_update/#4})

What about existing systems?
* Database technology developed for single-node systems; gold standard: serializability
	* Serializability is not actually highly available; give an example!
	* Much of the database literature presumes serializability and does not consider high availability.
	* For the literature that doesn't presume serializability, it's not presented in a HA context?
* Consequence: traditional systems are not optimized for high availability
	* Consider two-phase locking in a distributed environment
	* As we will see (forward reference to Evaluation), many existing transactional systems encounter similar difficulties
* Key question in this paper: What transactional semantics are highly available, and which aren't?
	* Side note: there are infinite incomparable consistency models (e.g., always return 2, always return 3, …)
	* Our goal is to unify distributed systems literature with ACID transaction model.
* Argument: Highly Available Transactional Systems (HATS) require rethinking existing models. This work is a first step.

In this sense, HATs are similar to RAID: optimizing for graceful
handling of a worst-case failure scenario improves average-case
performance.
