
\section{High Availability}
\label{sec:availability}




highly available algorithms guaranteeing
``always on'' operation are applicable in modern distributed
contexts. This is because, a high level, a highly available system
(which we formally define in Section~\ref{sec:availabilty}) guarantees
two useful properties: during network partitions, the system responds
to requests without blocking, and, during normal, partition-free
operation, it provides guaranteed one round-trip-time (RTT) latency.

In this section, we survey evidence regarding the prevalence of
network partitions as well as quantify the cost of round-trip times
over public cloud infrastructure. These results are independent of
algorithms employed and, while they reflect state-of-the-art
technology, they are



Why High Availability?
* Highly available system operation is often desirable; if you can contact one server, you can get a response!
* As a corrollary, an "AP" system is faster than a "CP" system; see "PACELC" by Abadi.
	* Look, here are some datapoints we recently measured on EC2. Network times are at least 10x faster, often much more for "AP"


* System model
	* Consider transactions that issue a series of reads and writes, then commit or abort
	* Each data item is located on one or more servers

* HA Definitions
	* Straw-man definition: every transaction will recieve a response
		* Problem: system that always aborts is trivially available!

* Replica-HA
	* (Handwaving in the outline) If a client can contact at least one replica for each data item in a transaction, then the transaction will eventually commit, even in the presence of infinite network partitions.

* Sticky-HA
	* If a client continues to contact the same set of replicas across transactions and the client can contact at least one replica for each data item in a transaciton, then the transaction will eventually commit, even in the presence of infinite network partitions.
	* We will use this definition when it comes to causality and read your writes guarantees!
	* Can be achieved by either "sticking" clients to the same set of servers or by having the client act as a replica itself, via caching 




* As an example of a real system going from "CP" to "AP", [PNUTS dropped per-record mastering](\url{http://developer.yahoo.com/blogs/ydn/posts/2010/06/sherpa_update/#4})

What about existing systems?
* Database technology developed for single-node systems; gold standard: serializability
	* Serializability is not actually highly available; give an example!
	* Much of the database literature presumes serializability and does not consider high availability.
	* For the literature that doesn't presume serializability, it's not presented in a HA context?
* Consequence: traditional systems are not optimized for high availability
	* Consider two-phase locking in a distributed environment
	* As we will see (forward reference to Evaluation), many existing transactional systems encounter similar difficulties
* Key question in this paper: What transactional semantics are highly available, and which aren't?
	* Side note: there are infinite incomparable consistency models (e.g., always return 2, always return 3, …)
	* Our goal is to unify distributed systems literature with ACID transaction model.
* Argument: Highly Available Transactional Systems (HATS) require rethinking existing models. This work is a first step.

In this sense, HATs are similar to RAID: optimizing for graceful
handling of a worst-case failure scenario improves average-case
performance.
