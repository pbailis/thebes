
\section{Related Work}

Although we are not aware of any related work specifically addressing
what properties are available in a highly available system, there are
several systems and related research results upon which we draw in
this work.

Weak consistency and high availability have been well
studied. Serializability has long been known to be
unachievable~\cite{davidson-survey} and Brewer's CAP Theorem has
attracted considerable attention~\cite{gilbert-cap}. The research has
begun to offer hints regarding the connection between ``weak
consistency'' and low latency~\cite{abadi-pacelc}, while several
studies~\cite{adya, ansicritique} examine weak isolation guarantees,
grounded in a single-server context. The connection between weak
consistency models in a distributed setting~\cite{calm, crdt} and high
availability is relatively unexplored. With the exception of a
preliminary workshop paper on this work~\cite{hat-hotos}---upon which
this work builds, providing impossibility results, a full
taxonomization, experimental and application analysis (basically all
but Section 4.1)---we believe this paper is the first to explore this
area.


There has been a recent resurgence of interest in distributed
multi-object semantics, both in academia~\cite{kraska-s3, granola,
  gstore, mdcc, redblue, cops, eiger, walter,calvin, swift} and
industry~\cite{megastore, orleans, spanner}. We can roughly categorize
related work into two categories: unavailable systems with stronger
semantics than HATs and highly available systems with weaker
semantics.

Unavailable, strongly consistent databases have a long tradition. As
discussed in Section~\ref{sec:motivation}, classic ACID databases
provide strong semantics but their lock-based and traditional
multi-versioned implementations are unavailable in the presence of
partitions~\cite{bernstein-concurrency, bernstein-book,
  gray-isolation}. Similar semantics are common in recent
work. Notably, Google's Spanner provides externally serializable
transactions. While Spanner fits Google's read-mostly workload well,
read/write transactions use two-phase locking (incurring at least one
round trip time over wide-area networks), are unavailable to clients
in a non-majority partition, and cannot read their
writes~\cite{spanner}. These latency and availability penalties are
fundamental to providing strong consistency. For users willing to
tolerate these costs, Spanner, or similar strongly consistent,
unavailable systems---including Calvin~\cite{calvin},
G-Store~\cite{gstore}, Gemini~\cite{redblue}, Granola~\cite{granola},
HBase~\cite{hbase}, MDCC~\cite{mdcc}, Megastore~\cite{megastore},
Orleans~\cite{orleans}, and Walter~\cite{walter}---are a reasonable
choice. With HATs, we seek an alternative set of semantics that are
still useful but do not violate requirements for high availability or
low latency.
