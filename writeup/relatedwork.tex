
\section{Related Work}
\label{sec:relatedwork}

We have discussed traditional mechanisms for distributed
coordination and several related systems in
Section~\ref{sec:eval-existing}, but, in this section, we further discuss
related work. In particular, we discuss related work on highly
available semantics, mechanisms for concurrency control, and
techniques for scalable distributed operations.

Weak consistency and high availability have been well
studied. Serializability has long been known to be
unachievable~\cite{davidson-survey} and Brewer's CAP Theorem has
attracted considerable attention~\cite{gilbert-cap}. Recent work on
PACELC expands CAP by considering connections between ``weak
consistency'' and low latency~\cite{abadi-pacelc}, while several
studies examine weak isolation guarantees~\cite{adya,
  ansicritique}. There are a wide range of coordination-avoiding
``optimistic replication'' strategies~\cite{optimistic} and several
recent efforts at further taxonomizing these strategies in light of
current practice and the proliferation of ``eventually consistent''
stores~\cite{bailis-ec, bernstein-survey} (notably, Bernstein and
Das~\cite{bernstein-survey} specifically highlight the importance of
stickiness~\cite{sessionguarantees, vogels-defs}---which we formalize
in Section~\ref{sec:sticky}).  Aside from our earlier workshop paper
discussing transactional availability, real-world ACID, and HAT RC and
I-CI~\cite{hat-hotos}---which this work expands with additional
semantics, algorithms, and analysis---we believe this paper is the
first to explore the connections between transactional semantics, data
consistency, and (Gilbert and Lynch~\cite{gilbert-cap}) availability.

%, linearizability, consistent snapshots, and recency
%bounds~\cite{ceri-mechanism, chen-mechanism}.

There has been a recent resurgence of interest in distributed
multi-object semantics, both in academia~\cite{kraska-s3, gstore,
  eiger, walter,calvin, swift} and industry~\cite{orleans,spanner}. As
discussed in Section~\ref{sec:modernacid}, classic ACID databases
provide strong semantics but their lock-based and traditional
multi-versioned implementations are unavailable in the presence of
partitions~\cite{bernstein-book, gray-isolation}. Notably, Google's
Spanner provides strong one-copy serializable transactions. While
Spanner is highly specialized for Google's read-heavy workload, it
relies on two-phase commit and two-phase locking for read/write
transactions~\cite{spanner}. As we have discussed, the penalties
associated with this design are fundamental to serializability. For
users willing to tolerate unavailability and increased latency,
Spanner, or similar ``strongly consistent''
systems~\cite{kemme-classification}---including Calvin~\cite{calvin},
G-Store~\cite{gstore}, HBase, HStore~\cite{hstore},
Orleans~\cite{orleans}, Postgres-R~\cite{kemme-thesis},
Walter~\cite{walter}, and a range of snapshot isolation
techniques~\cite{daudjee-session}---reasonable
choices.

With HATs, we seek an alternative set of transactional semantics that
are still useful but do not violate requirements for high availability
or low latency. Recent systems proposals such as Swift~\cite{swift},
Eiger~\cite{eiger}, and Bolt-on Causal Consistency~\cite{bolton}
provide transactional causal consistency guarantees with varying
availability and represent a new class of sticky HAT systems. There
are infinitely many HAT models (i.e., always reading value 1 is
incomparable with always returning value 2), but a recent report from
UT Austin shows that no model stronger than causal consistency is
achievable in a sticky highly available, \textit{one-way convergent}
system~\cite{cac}. This result is promising and complementary to our
results for general-purpose convergent data stores. Finally, Burkhardt
et al. have concurrently developed an axiomatic specification for
eventual consistency; their work-in-progress report contains alternate
formalism for several HAT guarantees~\cite{burkhardt-txns}.


% Ceri discuss distributed database update mechanisms with respect to
% linearizability, consistent snapshots (i.e., cut isolation), and
% recency bounds~\cite{ceri-mechanism}.  Chen and Pu also classify
% distributed replica maintenance from perspectives of
% linearizability,% recency guarantees, and replica
% divergence~\cite{chen-mechanism}.



%, with increasing interest in returning to transactional
%designs~\cite{spanner, walter, foundation-article, krikellas-bargain,
%  eiger}

