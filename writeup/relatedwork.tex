
\section{Related Work}

There is a growing body related work on highly available semantics,
mechanisms for concurrency control, and techniques for scalable
distributed operations.

Weak consistency and high availability have been well
studied. Serializability has long been known to be
unachievable~\cite{davidson-survey} and Brewer's CAP Theorem has
attracted considerable attention~\cite{gilbert-cap}. Recent work on
PACELC expands CAP by considering connections between ``weak
consistency'' and low latency~\cite{abadi-pacelc}, while several
studies examine weak isolation guarantees~\cite{adya,
  ansicritique}. However, as we have discussed, the connection between
weak, multi-object consistency models in a distributed setting and
high availability is relatively unexplored. With the exception of our
preliminary workshop paper discussing transactional availability and
HAT RC and I-CI~\cite{hat-hotos}--- which this work expands, providing
additional algorithms and impossibility results, a full
taxonomization, and experimental and application analysis---we believe
this paper is the first to explore this area. However, several studies
rigorously classify \textit{non-HAT} semantics: Wisemann et al. have
proposed a three parameter classification for one-copy serializable
database systems, performing a more extensive classification than our
overview in Section~\ref{sec:evaluation}, including eager database
replication techniques~\cite{kemme-classification}. Additionally
several earlier surveys consider distributed mechanisms for achieving
one-copy serializability~\cite{wisemann-survey}, linearizability,
consistent snapshots, and recency bounds~\cite{ceri-mechanism,
  chen-mechanism}.

There has been a recent resurgence of interest in distributed
multi-object semantics, both in academia~\cite{kraska-s3, gstore,
  mdcc, eiger, walter,calvin, swift} and industry~\cite{orleans,
  spanner}. As discussed in Section~\ref{sec:modernacid}, classic ACID
databases provide strong semantics but their lock-based and
traditional multi-versioned implementations are unavailable in the
presence of partitions~\cite{bernstein-book, gray-isolation}. Notably,
Google's Spanner provides strong one-copy serializable
transactions. While Spanner is highly specialized for Google's
read-heavy workload, it relies on two-phase commit and two-phase
locking for read/write transactions~\cite{spanner}. As we have
discussed, the latency penalties associated with this design choice
are fundamental to serializability. For users willing to tolerate
these costs, Spanner, or similar strongly consistent, unavailable
systems---including Calvin~\cite{calvin}, G-Store~\cite{gstore},
Generalized Snapshot Isolation~\cite{generalizedsnapshot}, HBase,
HStore~\cite{hstore}, MDCC~\cite{mdcc}, Orleans~\cite{orleans},
Postgres-R~\cite{kemme-thesis}, Walter~\cite{walter}, and a range of
snapshot isolation techniques~\cite{middleware-db, kemme-snapshot,
  daudjee-snapshot}---are a reasonable choice.

With HATs, we seek an alternative set of semantics that are still
useful but do not violate requirements for high availability or low
latency. Recent systems proposals such as Swift~\cite{swift},
Eiger~\cite{eiger}, and Bolt-on Causal Consistency~\cite{bolton}
provide transactional causal consistency guarantees with varying
availability and represent a new class of sticky HAT systems.

\pbnote{add TA stuff}


% Ceri discuss distributed database update mechanisms with respect to
% linearizability, consistent snapshots (i.e., cut isolation), and
% recency bounds~\cite{ceri-mechanism}.  Chen and Pu also classify
% distributed replica maintenance from perspectives of
% linearizability,% recency guarantees, and replica
% divergence~\cite{chen-mechanism}.



%, with increasing interest in returning to transactional
%designs~\cite{spanner, walter, foundation-article, krikellas-bargain,
%  eiger}

