
\section{Related Work}

Weak consistency and high availability have been well
studied. Serializability has long been known to be
unachievable~\cite{davidson-survey} and Brewer's CAP Theorem has
attracted considerable attention~\cite{gilbert-cap}. The research has
begun to offer hints regarding the connection between ``weak
consistency'' and low latency~\cite{abadi-pacelc}, while several
studies examine weak isolation guarantees~\cite{adya,
  ansicritique}. However, as we have discussed, the connection between
weak consistency models in a distributed setting~\cite{calm, crdt} and
high availability is relatively unexplored. With the exception of a
preliminary workshop paper introducing HATs~\cite{hat-hotos}--- which
this work expands, providing impossibility results, a full
taxonomization, and experimental and application analysis (basically
all but Section 4.1)---we believe this paper is the first to explore
this area. Nonetheless, several studies rigorously classify non-HAT
semantics: Wisemann et al. have proposed a three parameter
classification for one-copy serializable database systems, performing
a more extensive classification than our overview in
Section~\ref{sec:evaluation}, including eager database replication
techniques~\cite{kemme-classification}. Additionally several earlier
surveys discuss distributed mechanisms for achieving one-copy
serializability~\cite{wisemann-survey}, linearizability, consistent
snapshots, and recency bounds~\cite{ceri-mechanism, chen-mechanism}.

There has been a recent resurgence of interest in distributed
multi-object semantics, both in academia~\cite{kraska-s3, granola,
  gstore, mdcc, redblue, cops, eiger, walter,calvin, swift} and
industry~\cite{megastore, orleans, spanner}. As discussed in
Section~\ref{sec:modernacid}, classic ACID databases provide strong
semantics but their lock-based and traditional multi-versioned
implementations are unavailable in the presence of
partitions~\cite{bernstein-concurrency, bernstein-book,
  gray-isolation}. Notably, Google's Spanner provides strong one-copy
serializable transactions. While Spanner is highly specialized for
Google's read-heavy workload, it relies on two-phase commit and
two-phase locking for read/write transactions~\cite{spanner}. As we
have discussed, the latency penalties associated with this design
choice are fundamental to serializability. For users willing to
tolerate these costs, Spanner, or similar strongly consistent,
unavailable systems---including Calvin~\cite{calvin},
G-Store~\cite{gstore}, Gemini~\cite{redblue}, Generalized Snapshot
Isolation~\cite{generalizedsnapshot}, Granola~\cite{granola},
HBase~\cite{hbase}, MDCC~\cite{mdcc}, Megastore~\cite{megastore},
Orleans~\cite{orleans}, Postgres-R~\cite{kemme-thesis}, and
Walter~\cite{walter}---are a reasonable choice. We note that many
replication techniques target variants of Snapshot
Isolation~\cite{middleware-db}.

With HATs, we seek an alternative set of semantics that are
still useful but do not violate requirements for high availability or
low latency. Recent systems proposals such as Swift~\cite{swift},
Eiger~\cite{eiger}, and Bolt-on Causal Consistency~\cite{bolton}
provide transactional causal consistency guarantees with varying
availability and represent a new class of (sticky) HAT systems.







Middleware-based Database Replication: The Gaps Between Theory and Practice
Most uses Snapshot Isolation

% Ceri discuss distributed database update mechanisms with respect to
% linearizability, consistent snapshots (i.e., cut isolation), and
% recency bounds~\cite{ceri-mechanism}.  Chen and Pu also classify
% distributed replica maintenance from perspectives of
% linearizability,% recency guarantees, and replica
% divergence~\cite{chen-mechanism}.



, with increasing interest in returning to transactional
designs~\cite{spanner, walter, foundation-article, krikellas-bargain,
  eiger}

