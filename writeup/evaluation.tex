
\section{Analysis}
\label{sec:evaluation}

With a firm understanding of which of several ACID and distributed
consistency properties are available and not, we perform analysis of
their implications for existing algorithms, applications, and
deployments. Specifically:

\begin{enumerate}
\item We revisit traditional database concurrency control with a focus
  on coordination costs and on high availability.
\item We examine the properties required by a realistic OLTP
  application patterned on TPC-C.
\item We perform a brief experimental evaluation of HAT versus non-HAT
  properties on public cloud infrastructure.
\end{enumerate}

\subsection{Existing Algorithms}

As we have discussed, many existing database transaction and
concurrency control algorithms are not designed for high
availability. These algorithms often presume a single-server
deployment or the requirement for serializability. As a consequence,
traditional transaction processing systems are not well-optimized for
a HAT context.

We begin by considering several existing algorithms that provide
serializability but are not highly available. In summary, all
algorithms in this class require at least one round trip time (RTT)
before committing.

\begin{itemize}
\item \textbf{Two-phase locking} (ignoring deadlocks) requires $T$
  lock operations and 1 unlock operation for a transaction of length
  $T$. If all items are located on a single master server, this
  requires one RTT (for requests originating from servers other than
  the master). If items are located on $S$ ($S \leq T$) different
  servers, two-phase locking requires, at minimum, $2(S-1)$ RTTs---one
  for lock and another for unlock.

\item \textbf{Optimistic concurrency control} allows reads from any
  server but transactions must be checked by a scheduler before
  commit. With one scheduler per database, this requires one RTT for
  concurrency control. If schedulers are sharded by database
  partition, a multi-partition transaction may require contacting
  several different schedulers and, possibly, an atomic commit
  protocol, which becomes substantially more expensive.

\item \textbf{Deterministic transaction scheduling} is effecitvely the
  inverse of optimistic concurrency control: a scheduler first orders
  the transactions, then sends them to database replicas for in-order
  execution. These techniques, pioneered in the context of totally
  ordered atomic broadcast systems, have recently seen increased
  attention and effectively require communication overhead similar to
  optimistic concurrency control.

\item\textbf{Multiversion concurrency control} allows readers to read
  older versions of data items but, like OCC, must eventually be
  checked for write-write conflicts, requiring a scheduler.

\end{itemize}

There are several mechanisms that do not provide serializability (and
often provide HAT semantics) but which are not
available. Non-serializability can but does not necessarily imply high
availability and low latency:

\begin{itemize}
\item \textbf{Lock-based protocols} such as those used to provide weak
  isolation in Gray's original paper require at least one RTT to the a
  lock server (or, alternatively, a majority quorum). In the presence
  of failure, distributed locking can be challenging.
\item The \textbf{MDCC} commit protocol uses a Paxos variant and escrow
  transactions to provide Repeatable Read isolation with 1-2 RTTs. (TODO: CHECK)
\item \textbf{Parallel Snapshot Isolation} as developed by the Walter
  system provides (unavailable) snapshot isolation variant and, even
  when writes commute, requires 1 RTT to enforce read
  recency. However, PSI without recency requirements could be made
  highly available.
\item \textbf{Eiger} provides read-only and write-only transactions
  with item cut isolation and causal consistency. It relies on sticky
  high availability and groups within a datacenter with linearizable
  clusters. In the presence of server failure during write
  transactions, reads and writes may become unavailable. \textbf{COPS}
  provides similar semantics for read-only transactions but does not
  suffer from unavailability as it does not have a coordinator.
\item \textbf{Chan and Gray read-only transactions} use a similar
  algorithm to Eiger to provide read-only transactions with item cut
  isolation and causal consistency but coordinate on reads instead of
  writes.
\item \textbf{Bolt-on Causal Consistency} and \textbf{Swift} can
  provide read-only and write-only transactions with item cut
  isolation and causal consistency with sticky high availability and
  both rely on client-side caching.
\item \textbf{Bayou} provides several session guarantees but, without
  client-side caching, actually faces unavailability in the presence
  of failure. (TODO: CHECK)
\item \textbf{Brantner's S3 Database} uses caching to provide several
  guarantees such as XX and YY but is unavailable during failures
  (TODO: CHECK).
\end{itemize}

\subsection{Application Requirements}

Thus far, we have largely ignored the question of which applications
actually require HAT semantics versus traditional ACID guarantees. As
we summarized in Section~\ref{needed}, the main cost of providing HAT
updates is the inability to prevent Lost Update, Write Skew, and
provide recency bounds. In this Section, we attempt to understand when
these guarantees matter both abstractly and by the use of a
representative customer order transaction application patterned off of
TPC-C.

Recent work on the CALM Theorem, Lattice-based state update, and
Commutative and Replicated Data Types demonstrates that, if updates
logically commute, then they can be performed in different orders at
different locations in a distributed system. Accordingly, as long as
all writes are delivered to all replicas, then a system executing
monotonic logic with commutative operators should not suffer from
application-level consistency anomalies as a result of Lost Update or
Write Skew anomalies. However, applications with non-monotonic state
mutation will not, in general, be able to maintain application-level
consistency constraints. Similarly, applications requiring
write visibility between clients should opt for unavailability instead
of possibly missing a write visibility deadline.

Absent a necessary and sufficient result describing exactly which
application logics can be achieved with high availability and to make
these requirements more concrete, we consider an example application:
the TPC-C benchmark. Our goal is \textit{not} to provide a performance
analysis, as any HAT configuration will be non-compliant. Rather, we
examine the five queries contained in the benchmarks---ostensibly
representative of a standard transaction processing application---to
understand which are achievable in a HAT context.

TPC-C consists of five transactions, capturing operation of a
real-world warehouse, including sales, payments, and deliveries. We
analyze each transaction and its interactions below.

\begin{itemize}
\item \textbf{New-Order} places an order for a variable quantity of
  data items, updating warehouse stock as needed. The New-Order
  transaction selects a sales district, assigns the new order an order
  ID number, adjusts the remaining warehouse stock, and writes a
  placeholder entry for the pending order. There are two mutations in
  this transaction which we must be careful of.

  First, assigning an order ID number is challenging. We can easily
  assign a \textit{unique} order ID, by, say, concatenating the user
  ID with a timestamp and server ID. However, the TPC-C specification
  requires that order numbers be sequentially assigned, which we
  cannot guarantee in HATs due to Lost Update. Accordingly, while HATs
  cannot provide compliant TPC-C execution due to a non-monotonic
  counter increase, they can allow placement of new orders, which is a
  commutative and monotonic operation

  Second, decrementing inventory counts is a potentially risky
  operation as, without coordination, inventory counts may become
  negative. However, TPC-C automatically ``refills'' inventory counts
  (by 91 items) when there are 10 or fewer items
  remaining. Accordingly, even in the presence of concurrent new
  orders to the same item, the stock will never fall below
  zero. However, if the system propagates ``refill'' operations
  between replicas, then an $R$ replica system might end up with
  $91(R-1)$ more stock than a single-site implementation.

\item \textbf{Payment} updates running balances for warehouses,
  districts, and customer records as well as providing an audit
  trail. All balance operations can be made commutative, so the
  transaction can execute under HATs.

\item \textbf{Order-Status} is a read-only transaction and can be
  safely executed under HATs. The TPC-C specification does not require
  recency bounds but, in practice, they may be useful. If a client is
  sticky with her replica set, she will read her own writes, which may
  be sufficient.

\item \textbf{Delivery} represents the fulfillment of a pending order
  created by a New Order transaction: it deletes the order from the
  pending list, updates the customer's balance, updates the order's
  carrier ID and delivery time, and updates the customer balance. The
  challenge in delivery comes in deletion, which is non-monotonic: an
  item might be removed twice from the pending table. If delivery can
  be made idempotent---that is, if multiple Delivery transactions have
  the same effect as one---then multiple clients can execute the same
  Deliveries without fatal consequences.

  In practice, given that a physical shipment is likely only delivered
  once, it is unlikely that multiple agents will attempt a Delivery
  transaction for the same order simultaneously. However, in this
  case, we have simply moved the non-monotonicity to the real world:
  the carrier that picks up the package for an order must ensure that
  she is the only carrier who has done so. While we can program around
  the other operations in the order, HATs cannot--without the help of
  business-specific processes--serve Delivery transactions. That said,
  New Orders may be placed.

\item \textbf{Stock-Level} is another read-only transaction and can be
  executed without any non-HAT guarantees, with the same caveats
  regarding recency.
\end{itemize}

Throughout execution, TPC-C requires the maintainence of several
integrity constraints. For example, Consistency Condition 1 (3.3.2.1)
requires that each warehouse's sales count must reflect the sum of its
subordinate sales districts. This integrity constraint spans two
tables but, given the ability to update both tables atomically via
transactional atomicity, can be easily maintained. Consistency
Conditions 4 through 12 (3.3.2.4-12) can similarly be satisfied by
applying updates atomically across tables. Consistency Conditions 2
and 3 (3.3.2.2-3) concern order ID assignment and are more
problematic: effectively, each New Order placed by a customer in a
district must have an ID that is one the prior order in that
district. This requires preventing Lost Update within each district's
table. However, as we have seen, the required transactions can execute
without these two conditions.

In summary, many---but not all---TPC-C transactions are well-served by
HATs. The two problematic transactions---New Order and Payment---rely
on non-monotonic state updates. The former transaction can be modified
to ensure ID uniqueness at the expense of dense ID ordering, while the
latter is an inherently non-monotonic action requiring external
compensation or stronger consistency protocols. We expect that,
especially for read-dominated workloads found in many online services,
HAT guarantees may suffice for many applications. Finally, while TPC-C
is not subject to write skew anomalies, we note that many TPC-E
isolation tests, such as one that requires simultaneously modifying a
product description and its image, are satisfied by HATs.

\subsection{Experimental Costs}


2.) We ran YCSB on EC2 (and maybe TPC-C if we have time) with a few models:
	- Eventual Consistency
	- Eventual consistency with a "master" for updates--simulates the lower bound on a non-HAT system.
	- Transactional Atomicity and Read Committed
	
	-This is going to look a lot like the ping tests: much higher latency for "non-HATs."
	-Our Naive 2PL implementation bottlenecks extremely quickly.
	-On, say, 5 servers, can get hundreds of thousands of TPS

	-Not meant as an exhaustive study, but validates our intuitions. Coordination costs of non-HAT systems are unaccounted for, will only go up, and we haven't even talked about availability.
