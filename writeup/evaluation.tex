
\section{Analysis}
\label{sec:evaluation}

With a firm understanding of which of several ACID and distributed
consistency properties are available and not, we perform analysis of
their implications for existing algorithms, applications, and
deployments. Specifically:

\begin{enumerate}
\item We revisit traditional database concurrency control with a focus
  on coordination costs and on high availability.
\item We examine the properties required by a realistic OLTP
  application patterned on TPC-C.
\item We perform a brief experimental evaluation of HAT versus non-HAT
  properties on public cloud infrastructure.
\end{enumerate}

\subsection{Existing Algorithms}

As we have discussed, many existing database transaction and
concurrency control algorithms are not designed for high
availability. These algorithms often presume a single-server
deployment or the requirement for serializability. As a consequence,
traditional transaction processing systems are not well-optimized for
a HAT context.

 designed for either single-node deployments or
to achieve the are designed for

1.) There are a bunch of existing systems that aren't HATS but provide stronger semantics. What's their cost? (N.B. this is almost a related work survey here, but I think, if done right, it can be very instructive.)
Not HA, Serializable:
* Two-phase locking: for transaction size T, T+1 round-trips (directly to each database)
* Deterministic transaction scheduling: 1 round-trip per partition (to scheduler)
* Optimistic concurrency control: 1 round-trip per partition (to database cycle checker)
* MVCC: min 1 round-trip per partition (to partition timestamp manager)

Not HA, not serializable, but "transactional":
* MDCC: need to check, 1-2 round-trip times because uses Paxos variant
* Walter: 1 rount-trip time to each preferred site
* Eiger: Sticky-HA, assumes linearizable clusters, reads/writes stall in presence of coordinator failures
* Chan and Gray read-only transaction: reads stall in the presence of coordinator failure
* Bolt-on causal consistency: sticky-HA, could be used to implement TA+RR, uses client cache
* COPS: Sticky-HA, assumes linearizable clusters
* Bayou: not HA since client doesn't cache (pretty sure)

\subsection{Application Requirements}

2.) We ran YCSB on EC2 (and maybe TPC-C if we have time) with a few models:
	- Eventual Consistency
	- Eventual consistency with a "master" for updates--simulates the lower bound on a non-HAT system.
	- Transactional Atomicity and Read Committed
	
	-This is going to look a lot like the ping tests: much higher latency for "non-HATs."
	-Our Naive 2PL implementation bottlenecks extremely quickly.
	-On, say, 5 servers, can get hundreds of thousands of TPS

	-Not meant as an exhaustive study, but validates our intuitions. Coordination costs of non-HAT systems are unaccounted for, will only go up, and we haven't even talked about availability.

\subsection{Experimental Costs}



