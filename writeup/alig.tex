

For that, the model of the system has to be changed to include
clients, which cannot be mobile, but rather have to {\em stick} to one
server, something which is implicit in much of the previous work on
distributed systems.


This taxonomy allows us to weigh the costs and benefits of a wide
range of semantic guarantees, but which guarantees are worthwhile in
practice? We attempt to objectively study the virtues and limitations
of both classic and HAT systems by surveying practitioner accounts and
research literature, performing experimental analyses on modern cloud
infrastructure, and analyzing representative applications for their
semantic requirements. While our results are not definitive for all
applications, they suggest that HATs offer a one to three order of
magnitude latency decrease compared to traditional protocols and can
provide acceptable semantics for a wide range of programs, especially
those with monotonic logic and commutative updates~\cite{calm, blooml,
  crdt}. HAT systems can also enforce arbitrary foreign key
constraints on multi-item updates and sometimes provide limited
uniqueness guarantees. However, particularly for programs with
non-monotonic logic, HATs may fall short, requiring more
coordination-intensive, unavailable mechanisms. Concisely, we attempt
to understand \textit{what benefits do Highly Available Transactions
  offer and what is their cost?}
