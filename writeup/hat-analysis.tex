
\section{Highly Available Transactions}
\label{sec:hats}

\begin{quote}
``...one should not throw out the C so quickly, since there are real
  error scenarios where CAP does not apply and it seems like a bad
  tradeoff in many of the other situations.''---Michael
  Stonebraker~\cite{stonebraker2010errors}
\end{quote}

HAT systems---providing transactions with transactional availability
with high availability ($1$-available) or sticky high availability
($1$-sticky-available)---offer substantial latency and availability
benefits, yet they come with a cost to achievable semantics. In this
section, we delineate which of several ACID, distributed consistency,
and session consistency levels can be achieved with high availability
(e.g., transactional atomicity, variants of Repeatable Read isolation,
and many session guarantees) and sticky high availability (e.g.,
read-your-writes, PRAM, and causal consistency) and which cannot
(e.g., preventing Lost Update and Write Skew).  We present a full
summary of these results in Section~\ref{sec:hat-summary}.

When possible, we draw on existing properties and definitions from the
database and distributed systems literature, providing a brief,
informal explanation and example for each guarantee. In particular,
for our ACID guarantees, we draw largely on Atul Adya's
dissertation~\cite{adya}, Berenson et al.'s 1995 critique of the ANSI
SQL Specification~\cite{ansicritique}, and the ANSI SQL
specification~\cite{ansi-sql}. We provide a set of formal definitions
and semantics in our extended Technical Report~\cite{hat-tr}.


\subsection{Achievable HAT Semantics}

To begin, we present achievable semantics, offering proof-of-concept
algorithms to demonstrate feasibility. These guarantees are fairly
straightforward and have been introduced---albeit with greater brevity
and discussion elided---in a preliminary workshop
paper~\cite{hat-hotos}. In our examples, we exclusively consider read
and write operations. We denote a write of value $v$ to data item $d$
as $w_d(v)$ and a read from data item $d$ returning $v$ as $r_d(v)$
and assume that all data items have the null value, $\bot$, at
database initialization. Unless otherwise specified, all example
transactions commit.

\subsubsection{ACID Isolation Guarantees}

To begin, Adya's \textit{Read Uncommitted} isolation requires that
transactions are totally ordered and that writes within transactions
are ordered consistently with this order (prohibiting ``Dirty
Writes,'' or $G0$)~\cite{adya}. If two transactions write to the same
set of data items, then the final database state cannot contain the
``earlier'' transaction's writes to the data items. For example, in
the below example, $T_3$ should eventually only read $a=b=1$ or
$a=b=2$ but not $a=2, b=1$ or $a=1, b=2$:
\begin{align*}
\small
T_1 &: w_x(1)~w_y(1)
\\T_2 &: w_x(2)~w_y(2)
\\T_3 &: r_x(a)~r_y(b)\\[-2em]
\end{align*}
We will strengthen this property in later models, which will prevent,
for instance, reading $a=2, b=1$ at any time (not just
``eventually''). The difference between this total order on
transactions and the total order required by serializability is that
the Read Uncommitted total order is completely arbitrary (i.e., an
order must exist), whereas a serializable total order must be
equivalent to a serial execution. It is easily achieved via applying
the same logical timestamp to each update in a transaction and
applying a ``last writer wins'' conflict reconciliation policy at each
replica.

\textit{Read Committed} isolation requires that transactions do not
read uncommitted versions of data items (prohibiting both ``Dirty
Writes''---as above---and ``Dirty Reads'' phenomena; captured by
Adya's $G1a-c$, ANSI's $P1$, and ``broad'' $P1$ (2.2) from Berenson et
al.). For instance, in the example below, $T_3$ should never see
$a=1$, and, if $T_2$ aborts, $T_3$ should never read $a=3$:
\vspace{-.5em}
\begin{align*}
\small
T_1 &: w_x(1)~w_x(2)
\\T_2 &: w_x(3)\\
T_3 &: r_x(a)\\[-2em]
\end{align*}
It is fairly easy to prvent ``Dirty Reads'': if each transaction never
writes uncommitted data to the database, then transactions will never
read each others' dirty data. Clients can buffer their writes until
they commit, or, alternatively, can send them to servers, who will not
serve new writes until clients notify them that the writes have been
committed.

\textit{Repeatable Read} isolation is a contentious property. As
Berenson et al. discuss~\cite{ansicritique}, Gray's original
Repeatable Read lock-based implementation provides substantially
richer guarantees than those that are required by the ANSI SQL
specification~\cite{gray-isolation}. Gray~\cite{gray-isolation},
Berenson et al.~\cite{ansicritique}, and Adya~\cite{adya} all
interpret Repeatable Read as providing serializability for all
operations except predicate-based reads. However, in its description
of Repeatable Read, the ANSI SQL specification provides a useful
property which, although is not true to Gray's spirit of Repeatable
Read, is also commonly found in distributed consistency models: the
ANSI Repeatable Read guarantee requires that, along with observing
Read Committed isolation, each transaction will only read one version
of each data item that it did not itself produce (preventing ``Fuzzy
Read,'' or P2). In the example below, $T_3$ must read $a=1$:
\begin{align*}
\small
T_1 &: w_x(1)
\\T_2 &: w_x(2)
\\T_3 &: r_x(1)~r_x(a)
\end{align*}
This isolation property is a literal interpretation of the phrase
``Repeatable Read'': unless a transaction modifies a given data item,
the observable value of the data item should not change during the
transaction. By itself, this property is rather weak---we can always
read $\bot$ for each data item. However, when coupled with additional
properties, like transactional atomicity, it is stronger: the repeated
reads must obey additional ordering constraints. In fact, distributed
systems often describe a ``consistent snapshot'' across a set of
related events as a \textit{cut} across a set of participants or, in
our case, data items. Accordingly, to capture the notion that the
transaction should read from a (non-changing) consistent cut over data
items (and to disambiguate from the aforementioned stronger Repeatable
Read properties), we call ANSI SQL ``Repeatable Read'' (i.e.,
preventing P2) \textbf{Cut Isolation}. It is possible to satisfy cut
isolation with high availability by caching appropriate versions for
each of a transaction's reads or, alternatively, using
multi-versioning on each server and ensuring that each of a
transaction's successive reads return the same version of each data
item.

We can further consider two variants of Cut Isolation. The first is
relatively straightforward and provides a cut across individual data
items. We will call this \textit{Item Cut Isolation}. The second
provides a cut over individual data items plus logical ranges of data
items, or predicate-based reads. This prevents the \textit{Phantom
  Problem}, whereby two successive predicate-based reads return
different data~\cite{gray-isolation}. We call predicate-based cut
isolation, which prevents Phantoms, \textit{Predicate Cut
  Isolation}.\footnote{Oracle provides an isolation level called
  ``Statement Level Read Consistency''; this is analogous to Predicate
  Cut Isolation at the level of a single operation within a
  multi-operation transaction. We do not consider this isolation model
  here as it is subsumed by standard Snapshot Isolation and we believe
  our discussion is easily extended to incorporate it.}

\subsubsection{ACID Atomicity Guarantees}

Transactional atomicity (TA) is core to ACID guarantees. Although, at
least by the ACID acronym, it is not an ``isolation'' property,
transactional atomicity restricts transactions' ability to view the
effects of partially completed transactions. Under transactional
atomicity, within each transaction, either all effects of another
transaction are observed, or none are (equivalently, once some of the
effects of a transaction are observed, all effects are observed). This
is a strictly stronger guarantee than Read Committed isolation: in
Read Committed, we can read a subset of another transaction's
committed writes whereas, with TA, we must read all or none of
them. Together with item cut isolation, TA prevents Read Skew
anomalies (Berenson et al.'s A5A~\cite{ansicritique}). As an example
of TA, $T_2$ must observe $b=c=1$ (or later versions for each key):
\vspace{-.5em}
\begin{align*}
\small
T_1 &: w_x(1)~w_y(1)~w_z(1)
\\T_2 &: r_x(a)~r_y(1)~r_x(b)~r_z(c)~\\[-1.5em]
\end{align*}
$T_2$ can also observe $a=\bot$, $a=1$, or a later version of
$a$. Notably, TA requires Read Committed isolation: observing all
effects of a transaction implicitly requires observing the final
(committed) effects of a transaction as well.

Perhaps perplexingly, discussions of TA are absent from existing
discussions of weak isolation. This is perhaps again due to the single-node
context in which prior work was developed: on a single server (or a
fully replicated database), TA is near-trivial to achieve via
lightweight locking and/or local concurrency control over data
items. In contrast, in a distributed environment, TA over arbitrary
groups of non-colocated items is considerably more difficult to
achieve with high availability: servers need to know when all of a
transaction's updates are present on their respective
replicas. However, replicas do not need to agree on when to reveal a
new value to clients, which would require consensus; accordingly, TA
only requires reliable broadcast (with some additional metadata for
clients) and is achievable in a HAT system. We omit a full discussion
of the algorithm and instead refer the reader to the extended version
of this paper~\cite{hat-tr}.

\subsubsection{Session Guarantees}

Our models have not yet considered interactions across transactions:
we have not guaranteed any ordering or continuity between transactions
(other than that there exists some arbitrary ordering). However, in
the distributed systems context, many useful \textit{safety}
guarantees span multiple application actions. In particular,
\textit{session guarantees} are used to describe guarantees across
groups of operations (here, transactions) issued by the same logical
user~\cite{sessionguarantees}. A \textit{session} is an application
context that consists of logical state that should persist between
transactions. For example, on a social networking site, all of a
user's transactions submitted between ``log in'' and ``log out''
operations might form a session.

There are several guarantees that we can make with high availability:

\vspace{.5em}\noindent\textit{{Monotonic reads}} requires that,
with a session, subsequent reads to a given object ``never return any
previous values''; reads from each item progress forward in a total
order. The ordering of reads should respect any total ordering on
transactions.

\vspace{.5em}\noindent\textit{{Monotonic writes}} requires that
each session's writes be serialized. Any order on transactions should
also respect the order in which each session submitted transactions.

\vspace{.5em}\noindent\textit{{Writes Follow Reads}} requires that, if
a session observes an effect of transaction $T_2$ and subsequently
commits transaction $T_2$, then another session can only observe
effects of $T_2$ if it can also observe $T_1$'s (or later values for
$T_1$'s effects). Any order on transactions should respect the
reads-from order.\vspace{.5em}

We can achieve the above guarantees by forcing servers to wait to
reveal new writes until their respective dependencies are fulfilled on
all replicas. This mechanism effectively requires that all clients
read from a globally agreed upon lower bound on the versions written
to the system. This is highly available as a client will never block
due to inability to find a server with a sufficiently up-to-date
version of a data item. However, it does not imply that transactions
will read their own writes or, in the presence of partitions, make
forward progress through the version history. The problem is that, if
a server becomes partitioned, under the highly available model, we
must handle the possibility that an unfortunate client will be forced
to issue her next requests against the partitioned server!

The solution to this conundrum is to give up high availability in
favor of sticky availability. Sticky availability permits three
additional models, which we first define and then prove are
unachievable in a generic highly available system:

\vspace{.5em}\noindent\textit{{Read your writes}} requires
that whenever a client reads a given data item after updating it, the
read returns the updated value (or a value with a higher ID).

\vspace{.5em}\noindent\textit{{PRAM}} (Pipelined Random Access
Memory) provides the illusion of serializing each session's operations
and is the combination of monotonic reads, monotonic writes, and read
your writes~\cite{herlihy-art}.

\vspace{.5em}\noindent\textit{{Causal
    consistency}}~\cite{causalmemory} is the combination of all of the
session guarantees~\cite{sessiontocausal} (alternatively, PRAM with
writes-follow-reads) and is also referred to as PL-2L
isolation~\cite{adya}).\vspace{.5em}

Read your writes is unavailable in a highly available system. Consider
a client that executes the following two transactions in succession:
\vspace{-.5em}
\begin{align*}
\small
T_1 &: w_x(1)
\\T_2 &: r_x(a)
\end{align*}
If the client executes $T_1$ against a server that is partitioned from
the rest of the other servers, the server must allow $T_1$ to
commit. If the client executes $T_2$ against the same (partitioned)
server, then it will be able to read its writes. However, if the
network topology shifts and the client can only contact a different
server which is partitioned from the server that executed $T_1$, then
the client will be inable to read its own writes and the system will
have to either stall indefinitely to allow the client to read her
writes (violating transactional availability) or will have to
sacrifice read your writes guarantees. Accordingly, read your writes
requires stickiness, and, because causal consistency and PRAM require
read your writes, they also require stickiness. Read your writes is
``free'' in a sticky system, while the remaining causality and PRAM
guarantees can be accomplished with the prior algorithms.

\subsubsection{Additional Guarantees}

\noindent{\textbf{Consistency}} A HAT system can make limited
consistency guarantees. It can often execute commutative and logically
monotonic~\cite{calm} operations without the risk of invalidating
(also monotonic) application-level integrity constraints. Our goal in
this paper is not to sketch the entire space of consistency models
that are achievable (see Section~\ref{sec:futurework}. We specifically
evaluate TPC-C transaction semantics under HAT consistency guarantees
in Section~\ref{sec:evaluation}.

\vspace{.5em}\noindent{\textbf{Durability}} As we briefly discussed in
Section~\ref{sec:availability}, a client requiring that its
transactions' effects survive $F$ server faults requires at least
($F+1$)-availability.

\vspace{.5em}\noindent{\textbf{Convergence}} To require that the
system propagates writes between replicas, we can require convergence,
or eventual consistency for each data item: in the absence of new
mutations to a data item, in the absence of partitions, all servers
should eventually agree on the value for each item. This is typically
accomplished by any number of anti-entropy protocols, which
periodically update neighboring servers with the latest value for each
data item~\cite{antientropy}. Establishing a final value is related to
determining a total order on transaction updates, as in Read
Uncommitted.

\subsection{Unachievable HAT Semantics}
\label{sec:unachievable-hat}

While there are infinitely many HAT models
(Section~\ref{sec:futurework}), at this point, we have largely
exhausted the range of achievable, previously defined (and useful) HAT
semantics. Before summarizing our possibility results, we will present
impossibility results for HATs, also defined in terms of previously
identified isolation and consistency anomalies. Perhaps most notably,
it is impossible to prevent Lost Update or Write Skew in a HAT system.

\subsubsection{Unachievable ACID Isolation}

In this section, we demonstrate that preventing Lost Update and Write
Skew---and therefore providing Snapshot Isolation, Repeatable Read,
and one-copy serializability---requires unavailability.

In the words of Berenson et al., \textit{Lost Update} occurs when one
transaction $T1$ reads a given data item, a second transaction $T2$
updates the same data item, then $T1$ modifies the data item based on
its original read of the data item, ``missing'' or ``losing'' $T2$'s
newer update. Consider a database containing only the following
transactions:
\begin{align*}
\small
T_1 &: r_x(a) w_x(a+2)
\\T_2 &: w_x(2)
\end{align*}
If $T_1$ reads $a=1$ but $T_2$'s write to $x$ precedes $T_1$'s write
operation, then the database will end up with $a=3$, a state that
could not have resulted in a serial execution due to from $T_2$'s
``Lost Update.''

It is impossible to prevent Lost Update in a highly available
environment. Consider two clients who submit the following $T_1$ and
$T_2$ on opposite sides of a network partition:
\begin{align*}
\small
T_1 &: r_x(100)~w_x(100+20=120)
\\T_2 &: r_x(100)~w_x(100+30=130)
\end{align*}
The final state of the database will converge to an execution that
could not have resulted from a serial excecution.\footnote{In this
  example, we assume that, as is standard in modern databases,
  replicas accept values as they are written (i.e., register
  semantics). This particular example could be made serializable via
  the use of commutative updates (Section~\ref{sec:eval}) but, as we
  show in the extended version of the paper~\cite{hat-tr}, the problem
  persists in the general case.} To prevent this from happening,
  either $T_1$ or $T_2$ should not have committed. Each client's
  respective server might try to detect that another write occurred,
  but this requires knowing the version of the latest write to $x$. In
  our example, this reduces to a requirement for linearizability,
  which is, via Gilbert and Lynch's proof of the CAP Theorem, provably
  unachievable with high availabilty~\cite{gilbert-cap}.

\textbf{Write Skew} is a generalization of Lost Update to multiple
keys. It occurs when one transaction $T1$ reads a given data item $x$,
a second transaction $T2$ reads a different data item $y$, then $T1$
writes to $y$ and commits and $T2$ writes to $x$ and commits. As an
example of Write Skew, consider the following two transactions:
\begin{align*}
\small
T_1 &: r_y(0)~w_x(1)
\\T_2 &: r_x(0)~w_y(1)
\end{align*}
As Berenson et al. describe, if there was an integrity constraint
between $x$ and $y$ such that only one of $x$ or $y$ should have value
$1$ at any given time, then this write skew would lead to a
non-serializable execution. Write skew in particular is a somewhat
esoteric anomaly---for example, it does not appear in
TPC-C~\cite{snapshot-serializable}---but, as a generalization of Lost
Update, it is also unavailable to HATs.

The inability to prevent Lost Update means that Consistent Read,
Snapshot Isolation, and Cursor Stability guarantees are all
unavailable. The inability to prevent Lost Update or Write Skew means
that Repeatable Read and One-Copy Serializability guarantees are
unavailable.

\subsubsection{Unavailable Recency Guarantees}

Data storage systems make various recency guarantees on reads of data
items. As we have discussed, one of the most famous is
linearizability~\cite{herlihy-art}, which states that reads will
return the last completed write to a data item, and there are several
other (weaker) variants such as safe and regular register
semantics. When applied to transactional semantics, the combination of
one-copy serializability and linearizability is called \textit{strong
  (or strict) one-copy serializability}~\cite{adya} (e.g., Google's
Spanner~\cite{spanner}). It is also common, particularly in systems
that allow reading from masters and slaves, to provide a guarantee
such as ``read a version that is no more than five seconds out of
date'' or similar. Unfortunately, an indefinitely long partition can
force a system to violate any recency bound, so recency bounds are
unavailable to HAT systems.

\subsection{Summary}
\label{sec:hat-summary}

We summarize our results in Table~\ref{table:hatcompared}. A wide
range of isolation levels are achievable in a HAT systems, including
transactional atomicity, cut isolation, and several session
guarantees. With sticky availability, a system can achieve read your
writes guarantees and PRAM and causal consistency. However, many other
prominent models, such as Snapshot Isolation, One-Copy
Serializability, and Strict Serializability cannot be achieved due to
the inability to prevent Lost Update and Write Skew phenomena.

We illustrate the hierarchy of available, sticky available, and
unavailable consistency models we have discussed in
Figure~\ref{fig:hatcompared}. Many models are simultaneously
achievable, but we find several particularly compelling. If we combine
all sticky-HAT guarantees, we have transactional, causal snapshot
reads. If we combine TA and P-CI, we have transactional snapshot
reads. We can achieve RC, MR, and RYW by simply sticking clients to
servers. And we can also combine unavailable models---for example, an
unavailable system might provide PRAM and One-Copy
Serializability~\cite{daudjee-session}.

To the best of our knowledge, this is the first unification of
database ACID, distributed consistency, and session guarantee
models. Perhaps satisfactorily, strong one-copy serializability
subsumes all other models, while considering the (large) power set of
all compatible models (e.g., the diagram depicts 96 possible highly available
combinations) hints at the vast expanse of consistency models found in
the literature. This taxonimization is \textit{not} exhaustive, but we
believe it lends substantial clarity into the relationships between a
large subset of the prominent ACID and distributed consistency
models. Additional read/write transaction semantics that we have
omitted should be easily classifiable based on the available
primitives and unavailable anomalies we have already discussed.

 \newcommand{\lostupdate}{$^\dagger$}
 \newcommand{\rwskew}{$^\ddagger$}
 \newcommand{\linearizable}{$^\oplus$}

\begin{table}[t!]
\begin{tabular}{| c | p{6cm} | }\hline
HA & Transactional Atomicity (TA), Read Uncommitted (RU), Read
Committed (RC), Item Cut Isolation (P-CI), Predicate Cut Isolation
(P-CI), Monotonic Reads (MR), Monotonic Writes (MW), Writes Follow
Reads (WFR)\\\hline Sticky-HA & Read Your Writes (RYW), PRAM,
Causal\\\hline Unavailable & Cursor Stability (CS)\lostupdate,
Snapshot Isolation (SI)\lostupdate, Repeatable Read
(RR)\lostupdate\rwskew, One-Copy Serializability
(1SR)\lostupdate\rwskew, Recency\linearizable, Safe\linearizable,
Regular\linearizable, Strong 1SR\lostupdate\rwskew\linearizable
\\\hline
\end{tabular}
\caption{Summary of highly available, sticky highly available, and
  unavailable models considered in this paper. Unavailable models are
  labeled by cause of unavailability: preventing lost
  update\lostupdate, preventing write skew\rwskew, and requiring
  recency guarantees\linearizable.}
\label{table:hatcompared}
\end{table}

\begin{figure}[t!]
\centering
\begin{tikzpicture}[scale=0.8]
  \tikzstyle{sticky}=[rectangle,draw=blue!50,fill=blue!20,thick]
  \tikzstyle{noha}=[ellipse,draw=red!50,fill=red!20,thick, inner sep=0pt,minimum size=12pt]

  \tikzstyle{every node}=[font=\small]

 \node[draw=none,fill=none] (ici) at (1.2, 0) {I-CI};
 \node[draw=none,fill=none] (pci) at (1.65, 1.2) {P-CI};
 \node[draw=none,fill=none] (rc) at (-1.2, .8) {RC};
 \node[draw=none,fill=none] (ru) at (-1.2, 0) {RU};

 \node[draw=none,fill=none] (ta) at (-.2, 1.3) {TA};
 \node[draw=none,fill=none] (mr) at (2.4, 0) {MR};
 \node[draw=none,fill=none] (mw) at (3.6, 0) {MW};
 \node[draw=none,fill=none] (wfr) at (6.3,0) {WFR};
 \node at (5,0) [sticky] (ryw) {RYW};

 \node[noha](recency) at (7.7, 0) {recency};
 \node[noha](safe) at (7.7, 1) {safe};
 \node[noha](regular) at (7.7, 2) {regular};
 \node[noha](linearizable) at (7.7, 3) {linearizable};
 \node at (3.6, 2) [sticky] (causal) {causal};
 \node at (3.6, 1) [sticky] (pram) {PRAM};
 \node[noha] (cs) at (-1.2, 1.8) {CS};
 \node[noha] (rr) at (-.2, 2.7) {RR};
 \node[noha] (si) at (2.2, 2.4) {SI};
 \node[noha] (1sr) at (1.2, 3.2) {1SR};
 \node[noha] (ssr) at (3.6, 3.6) {Strong-1SR};

 \draw [->, red] (recency) -- (safe);
 \draw [->, red] (safe) -- (regular);
 \draw [->, red] (regular) -- (linearizable);
 \draw [->, red] (linearizable) -- (ssr);
 \draw [->, red] (1sr) -- (ssr);
 
 \draw [->] (ru) -- (rc);
 \draw [->] (rc) -- (ta);
 \draw [->] (ici) -- (pci);

 \draw [->, blue] (mr) -- (pram);
 \draw [->, blue] (mw) -- (pram);
 \draw [->, blue] (wfr) -- (causal);
 \draw [->, blue] (ryw) -- (pram);
 \draw [->, blue] (pram) -- (causal);

 %\draw[snake=coil, segment aspect=0, segment amplitude=.75pt, segment length=2pt] (ru) -- (mr);
 %\draw[snake=coil, segment aspect=0, segment amplitude=.75pt, segment length=2pt] (rc) -- (ta);
 %\draw[snake=coil, segment aspect=0, segment amplitude=.75pt, segment length=2pt] (ici) -- (ta);
 %\draw[snake=coil, segment aspect=0, segment amplitude=.75pt, segment length=2pt] (pci) -- (ta);
 %\draw[snake=coil, segment aspect=0, segment amplitude=.75pt, segment length=2pt] (rc) -- (mr);
 %\draw[snake=coil, segment aspect=0, segment amplitude=.75pt, segment length=2pt] (pci) -- (mr);
 %\draw[snake=coil, segment aspect=0, segment amplitude=.75pt, segment length=2pt] (ici) -- (mr);
 %\draw[snake=coil, segment aspect=0, segment amplitude=.75pt, segment length=2pt] (ta) -- (ru);
 %\draw[snake=coil, blue, segment aspect=0, segment amplitude=.75pt, segment length=2pt] (ru) -- (causal);
 %\draw[snake=coil, segment aspect=0, segment amplitude=.75pt, segment length=2pt] (mr) -- (mw);
 %\draw[snake=coil, segment aspect=0, segment amplitude=.75pt, segment length=2pt] (wfr) -- (mw);
 %\draw[snake=coil, blue, segment aspect=0, segment amplitude=.75pt, segment length=2pt] (wfr) -- (ryw);

 \draw [->, red] (rc) -- (cs);
 \draw [->, red] (cs) -- (rr);
 \draw [->, red] (pci) -- (si);
 \draw [->, red] (ici) -- (rr);
 \draw [->, red] (rr) -- (1sr);
 \draw [->, red] (si) -- (1sr);
 \draw [->, red] (ta) -- (si);
 \draw [->, red] (ta) -- (rr);
 \draw [->, red] (causal) -- (ssr);

\end{tikzpicture}
\label{fig:hat-order}
\caption{HAT, sticky HAT (in boxes), and unavailable models (circled)
  from Table~\protect\ref{table:hatcompared} compared
  graphically. Directed edges represent ordering by model
  strength. Models that do not share a common ancestor can be
  simultaneously achieved, and the resulting availability is that of
  the weakest available model in the combination.}
\label{fig:hatcompared}
\end{figure}


\subsection{Discussion}
\label{sec:discussion}

In this section, we discuss several subleties in our results,
specifically addressing model composition, transactional atomicity
versus linearizability, and stickiness requirements.

\vspace{.5em}\noindent\textbf{Model Composition} Choosing between
combinations of compatible guarantees requires care. Consider the
following transactions:
\begin{align*}
\small
T_1 &: w_x(1)~w_y(1)
\\T_2 &: w_x(2)~w_y(2)
\\T_3 &: r_x(a)~r_y(b)
\end{align*}
If we want to guarantee both cut isolation and transactional atomicity
and the system only executes $T_1$, $T_2$, and $T_3$, then $T_3$ needs
to read $a=b=\bot$, $a=b=1$, or $a=b=2$. This means that either the
implementation should frequently return $\bot$ (definitely undesirable
and possibly non-convergent), keep multiple versions of each data item
(necessitating potentially complicated distributed garbage
collection), or use pre-declared read sets to fetch a consistent cut
of keys before each transaction begins to execute. Using client-side
caching can alleviate some of these challenges~\cite{bolton, swift},
but then the system becomes sticky high available.

Composition cost also varies by property. For instance, Charron-Bost
has proven that, to capture causality between $N$ communicating
processes, standard vector-based approaches face an upper bound of
$O(N)$ storage per write~\cite{charron-bost}. This means that, with
$100K$ clients, each write might be accompanied by $100K$ timestamps
per vector. This is difficult to scale. By compromising on
availability (e.g., treating a datacenter as a linearizable cluster),
this overhead can be reduced~\cite{cops, eiger}, but it is much
cheaper to provide, say, read your writes, than full causal
consistency.

\vspace{.5em}\noindent\textbf{Linearizability and Transactional
  Atomicity} The relationship between linearizability and
transactional atomicity is non-obvious. One might attempt to find a
mapping between the two: transactional atomicity dictates that writes
to multiple keys across multiple servers are made visible to readers
all at once, while linearizability dictates that writes to a single
key on multiple servers are made visible to all readers at once. The
difference is two-fold. First, in linearizable (and safe and regular)
systems, writes are made visible to all clients \textit{immediately}
after they finish. With transactional atomicity, there is no recency
guarantee. Second, in linearizabile systems, all clients see all
writes at the same time. With transactional atomicity, clients see
writes at different times depending on which replicas they contact. We
are not aware of an analogous model in the distributed systems
literature. Accordingly, despite apparent similarities, transactional
atomicity is much weaker than linearizability.

\vspace{.5em}\noindent\textbf{Visibility and Stickiness} Sticky
availability can result in much better write \textit{visibility}:
clients will be able to safely read writes more quickly in a sticky
available system. In the model we discussed, it is possible to achieve
several properties like monotonic reads in a highly available system
by waiting to reveal a write until all servers have seen it and its
relevant dependencies. However, this incurs severe visibility
penalties---new writes will not become visible to clients in the
presence of partitions. A client that does not want to guarantee
read-your-writes (due to the sticky availability requirement) may
still wish to read other clients' writes with timeliness. Accordingly,
a system may forgo high availability in order to provide better
visibility. We also note that a system may wish to provide stickiness
in the absence of partitions, but, in the presence of partitions,
``lose'' session guarantees by having clients reconnect to an
available replica. This latter strategy provides session guarantees in
the absence of partitions but, to provide availability, sacrifices
them in the event of partitions.

