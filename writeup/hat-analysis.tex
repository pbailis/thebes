
\section{Highly Available Transactions}
\label{sec:hats}

\begin{quote}
``...one should not throw out the C so quickly, since there are real
  error scenarios where CAP does not apply and it seems like a bad
  tradeoff in many of the other situations.''---Michael
  Stonebraker~\cite{stonebraker2010errors}
\end{quote}

HAT systems offer substantial latency and availability benefits, yet
they come with a cost. In the distributed systems community, there is
a host of results that taxonomize the relationships between various
distributed consistency models. Perhaps most famously, Brewer's CAP
Theorem---as formally proven by Gilbert and
Lynch~\cite{needed}---states that it is impossible to maintain
linearizability (``C,'' or ``consistency''---albeit poorly named), or
the ability to read the last completed write, and replica availability
in the presence of failures~\cite{needed}. While the CAP Theorem has
has roots in decades of distributed system designs~\cite{needed}, it
has had large scale impact on the design and development of
distributed systems, particularly for Internet
services---unsurprisingly, the context in which Brewer formulated his
finding.

\subsection{ACID and Modern Databases}

While linearizability is a useful property to consider, there are
several other guarantees that are perhaps more relevant to a database
system. In particular, the database tradition of providing ACID
guarantees: atomicity, consistency, isolation, and durability are,
with few exceptions, not addressed by the distributed systems
literature. There are several factors behind this occurrence, the
least of which is that distributed systems rarely consider
multi-object guarantees, whereas ACID transactions are frequently used
to provide semantic guarantees across multiple operations on multiple
objects. While linearizability (unfortunately often called
``atomicity'') ``composes'' across multiple objects, a linearizable
system does not provide ACID ``atomicity'' (which we will call
``transactional atomicity'' for clarity): just because two writes are
visible to all writers immediately after they complete does not, on
its own, provide any guarantees about the mutual visibility of the two
writes.

Database designers and researchers have long realized that
serializability---the gold standard of ACID isolation, which
guarantees application-level consistency---is not achievable in a
highly available system. Indeed, decades of research on
semantics-based concurrency control and coordination-reducing
techniques such as escrow transactions and sagas allow serializability
in the presence of partial system knowledge or otherwise reduce
coordination cost. However, the majority of these algorithms are
intended to preserve application-level consistency, which, for
arbitrary applications and arbitrary sequence of operations, cannot be
ensured with high availability in the presence of partitions.

Of course, database systems offer a range of ACID properties besides
serializability. Even on a single-node system, the coordination
penalties associated with ensuring equivalence with a serial execution
can be severe and manifest themselves in the form of concurrency
control contention (and, subsequently, performance degradation,
scalability limitations, and, often, external aborts like
deadlocks). Instead, databases offer a host of so-called \textit{weak
  isolation} models that allow varying restrictions on the space of
schedules that are allowable by the system. None of these weak
isolation models guarantees serializability, but the benefits of these
models can outweigh the costs of possible application-level
consistency anomalies that arise from their use.

To understand the prevalence of these weak isolation models, we
surveyed the default and maximum isolation models provided by 18
relational databases, often claiming to provide ``ACID'' or ``NewSQL''
functionality. As shown in Table~\ref{table:existing}, only three out
of 18 databases provides serializability by default, and at least
eight do not provide serializability as an option at all. This is
particularly surprising when we consider the widespread deployment of
many of these databases, like Oracle 11g, which are known to power
major businesses and product functionality. We might view the use of
weak consistency is somewhat of an ``arms race'': anecdotally, one
major RDBMS vendor was forced to lower their default isolation level
in order to compare favorably in out-of-the-box performance
comparisons with its major competitor, which had already lowered its
default setting. Applications can guard against inconsistency in a
weakly isolated environment by performing their own locking external
to the database or by performing ad-hoc concurrency control within the
database. However, this appears error-prone and, without further
evidence, it remains somewhat puzzling why this is indeed favorable to
providing serializability with in the database itself.

\begin{table}
\begin{small}
\begin{tabular}{|l|c|c|}
\hline
Database & Default & Maximum\\\hline
Actian Ingres 10.0/10S~\cite{actian-docs} & S & S\\
Aerospike~\cite{aerospike-docs} & RC & RC\\
Akiban Persistit~\cite{akiban-docs} & SI & SI\\
Clustrix CLX 4100~\cite{clustrix-docs} & RR & ?\\
Greenplum 4.1~\cite{greenplum-docs} & RC & S \\
IBM DB2 10 for z/OS~\cite{db2-docs} & CS & S\\
IBM Informix 11.50~\cite{informix-docs} & Depends & RR\\
MySQL 5.6~\cite{innodb-docs} & RR & S \\
MemSQL 1b~\cite{memsql-docs} & RC & RC\\
MS SQL Server 2012~\cite{ms-sql-docs} & RC & S \\
NuoDB~\cite{nuodb-docs} & CR & CR\\
Oracle 11g~\cite{oracle-docs} & RC & SI\\
Oracle Berkeley DB~\cite{bdb-reg-docs} & S & S\\
Oracle Berkeley DB JE~\cite{bdb-je-docs} & RR & S\\
Postgres 9.2.2~\cite{postgres-docs} & RC & S\\
SAP HANA~\cite{hana-docs} & RC & SI\\
ScaleDB 1.02~\cite{scaledb-docs} & RC & RC\\
VoltDB~\cite{voltdb-docs} & S & S\\
\hline
\multicolumn{3}{|p{7cm}|}{{\footnotesize RC: read committed, RR: repeatable read, SI: snapshot isolation, S: serializability, CS: cursor stability, CR: consistent read}}\\\hline

\end{tabular}
\caption{Default and maximum isolation levels for ACID and NewSQL
  databases as of January 2013 (from
  \protect\cite{needed-hotos}).}\vspace{-1.5em}
\label{table:existing}
\end{small}
\end{table}

Given that these transactional models are frequently used, our
inability to provide serializability to provide general-purpose HATs
appears not to be a show-stopper. If applications cope with
inconsistency or, on a single node server, have already decided that
the benefits of weak isolation outweigh potential application
inconsistencies, then perhaps, in a highly available environment, some
may make a similar decision given the appropriate choice of semantic
guarantees.

The primary challenge in providing weak isolation in a HAT context is
that, unfortunately, it is unclear \textit{which} isolation guarantees
can be provided with high availability. Standard discussions of weak
isolation are presented in a single-node context---a possible
consequence of their original context: a lock-based (and therefore
unavailable, even in a distributed context) single-server
deployment~\cite{needed}. We are not aware of any prior literature
that provides guidance as to the relationship between weak isolation
and high availability. Prior work has examined the relationship
between serializability and high availability and weak isolation on a
single-server but never weak isolation and high availability taken
together. This apparent gap in the database literature is a possible
contributing factor to the often-apparent attitude that
general-purpose transactional semantics are either too expensive or
too lacking in availability guarantees to provide in modern
distributed databases---except under special circumstances, such as
operations over groups of co-located data items.

\subsection{Achievable HAT Semantics}

In this paper, we will describe which ACID semantics are achievable in
HATs as well as which are not. Our strategy is to explore successively
more powerful models until we reach a frontier of semantic guarantees
which are unavailable to HATs. To begin, we present achievable
semantics, offering proof-of-concept algorithms to demonstrate
feasibility. When possible, we draw on existing properties and
definitions from the database and distributed systems literature,
providing a brief, informal explanation of each guarantee and
illustrate it with an example. For our ACID guarantees, we draw
largely on Adya's thesis work on defining weak isolation
models~\cite{needed} and, where appropriate, supplement with
definitions and discussion from Berenson et al.'s SIGMOD 1995 critique
of the ANSI SQL Specification~\cite{needed} (hereafter, Berenson) as
well as the ANSI SQL specification~\cite{needed} (hereafter, ANSI)
itself. For other guarantees, such as session properties, we draw on
both the database and distributed systems literature. We provide a set
of formal definitions and semantics in the Appendix.

\subsubsection{ACID Isolation Guarantees}

To begin, Adya's \textit{Read Uncommitted} isolation requires that
transactions are totally ordered and that writes within transactions
are ordered consistently with this order (prohibiting ``Dirty
Writes,'' or $G0$)~\cite{adya}. If two transactions write to the same
set of data items, then the final database state cannot contain the
``earlier'' transaction's writes to the data items. For example, in
the below example, $T_3$ should eventually only read $a=b=1$ or
$a=b=2$ but not $a=2, b=1$:
\begin{align*}
\small
T_1 &: w_x(1)~w_y(1)
\\T_2 &: w_x(2)~w_y(2)
\\T_3 &: r_x(a)~r_y(b)\\[-2em]
\end{align*}
The difference between this total order on transactions and the total
order required by serializability is that the Read Uncommitted total
order is completely arbitrary (i.e., an order must exist), whereas a
serializable total order must be equivalent to a serial
execution. Interestingly, Read Uncommitted is a close analog with
eventually consistent systems's concept of replica convergence, in
which all replicas for a data item must end up with the same data
item. This convergence is usually achieved by picking a ``winning''
update---which requires ordering---or by maintaining a growing a set
of updates---whose cardinality implicitly defines an order (with
convergence achieved when all updates are in the set); we can see Read
Uncommitted as a cross-item convergence property. It is easily
achieved via applying the same logical timestamp to each update in a
transaction.

\textit{Read Committed} isolation requires that transactions do not
read uncommitted versions of data items (prohibiting both ``Dirty
Writes''---as above---and ``Dirty Reads'' phenomena; captured by
Adya's $G1\{a,b,c\}$, ANSI's $P1$, and Berenson's ``broad'' $P1$
(2.2)). For instance, in the example below, $T_3$ should never see
$a=1$, and, if $T_2$ aborts, $T_3$ should never read $a=3$:
\vspace{-.5em}
\begin{align*}
\small
T_1 &: w_x(1)~w_x(2)
\\T_2 &: w_x(3)\\
T_3 &: r_x(a)\\[-2em]
\end{align*}
It is fairly easy to prvent ``Dirty Reads'': if transactions never
write uncommitted data to the database, then transactions will never
read dirty data. Clients can buffer their writes until they commit,
or, alternatively, can send them to servers, who will not serve writes
until clients notify that they have been committed.

\textit{Repeatable Read} isolation is a contentious property. As
Berenson et al. point out, Gray's original Repeatable Read lock-based
implementation provides substantially richer guarantees than those
that are promised by the ANSI SQL specification. Gray, Berenson, and
Adya effectively preclude all serializability anomalies but the
Phantom Problem (which we have explicitly not yet considered and will
do so in Section~\ref{sec:unachievable-hat}). Nevertheless, the ANSI
SQL specification provides a useful property which, although is not
true to Gray's spirit of Repeatable Read, is used in distributed
consistency models. The ANSI Repeatable Read guarantee requires that,
along with observing Read Committed isolation, each transaction can
only read only version of each data item that it did not itself
produce. In the example below, $T_3$'s must see $a=1$:
\begin{align*}
\small
T_1 &: w_x(1)
\\T_2 &: w_x(2)
\\T_3 &: r_x(1)~r_x(a)
\end{align*}
This isolation property is much weaker than serializability, but it is
a good faith literal interpretation of the phrase ``Repeatable Read'':
transactions see their own writes. By itself, this property can be
implemented trivially---always read \texttt{null} for each
key. However, when coupled with additional properties, like
transactional atomicity, it becomes stronger: the repeated reads must
also obey some additional ordering properties. Distributed systems
describe a ``consistent snapshot'' across a set of related events as a
\textit{cut} across a set of participants or, in our case, data
items. Accordingly, to capture the notion that the transaction should
read from a consistent cut, and to disambiguate from the
aforementioned stronger Repeatable Read properties, we call ANSI SQL
Repeatable Read \textit{cut isolation}. In the absence of additional
guarantees, it is possible to satisfy cut isolation with high
availability by caching each transaction's reads or, alternatively,
using multi-versioning on the server and ensuring that each
transaction's successive reads return the same version of each data
item.

\subsubsection{ACID Atomicity Guarantees}

Transactional atomicity is core to ACID guarantees. Although, at least
by the acronym, not an ``isolation'' property, transactional atomicity
restricts transactions' ability to view the effects of partially
completed transactions: within a transactions, either all effects of
another transaction are observed, or none are (equivalently, once some
of the effects of a transaction are observed, all effects are
observed). This is a strictly stronger guarantee than Read Committed
isolation because, in Read Committed, we can read a subset of
committed writes whereas, with Transactional Atomicity, we must read
all of them. For example, if $T_1$ commits, then, given Transactional
Atomicity, $T_2$ must observe $b=c=1$ (or later versions):
\vspace{-.5em}
\begin{align*}
\small
T_1 &: w_x(1)~w_y(1)~w_z(1)
\\T_2 &: r_x(a)~r_y(1)~r_x(b)~r_z(c)~\\[-1.5em]
\end{align*}
$T_2$ can also observe $a=\bot$ or $a=1$.

Perhaps perplexingly, discussions of Transactional Atomicity are
absent from existing discussions of weak isolation levels. This is
perhaps again due to the single-node context in which prior work was
developed: on a single server, atomicity is near-trivial to achieve
via lightweight locking and/or local concurrency control over data
items. (In this spirit, multi-item atomicity is common in so-called
``entity groups,'' which logically represent groups of co-located
items.) In contrast, in a distributed environment, atomicity over
arbitrary groups of items is considerably more difficult to achieve
with high availability. The key challenge is that different servers
responsible for subsets of a transaction's updates must coordinate
regarding when to reveal their writes to a transaction---without
violating availability due to, say, locking, or a missing
transactional write.

(MORE HERE FROM MY NOTES)

Transactional atomicity is achievable in a HAT environment. To ensure
transactional atomicity, servers only need to know when all of a
transaction's updates are present on their respective servers but do
not need to know when all respective servers know when all of a
transaction's updates are present on their respective servers. This
subtle distinction corresponds to the difference between the
distributed systems concepts of consensus and uniform reliable
broadcast. The former requires explicit agreement across servers,
while the latter simply requires that, if one server reveals a write,
then all servers reveal the write.

(TODO: fix me--HotOS copy!)

Nonetheless, Coordinating transactional atomicity across replicas is
challenging. For example, if, as is common~\cite{readonly, eiger}, we
use a master to determine the correct set of updates to read, then the
system will not be available when clients are partitioned from the
master. Instead, we rely on decentralized \textit{background}
agreement to decide when to show new values, which can safely stall in
the presence of partitions. Each server keeps a known good value for
each data item $d$ it is assigned, $d_{good}$ and a set of pending
updates to $d$, $d_{pending}$. We use asynchronous agreement to move
updates from $d_{pending}$ to $d_{good}$: all of $d_{good}$'s
transactional siblings (or later values) are guaranteed present on all
replicas. Without aborts, this requires only one synchronous round
trip time (RTT).

\noindent\textit{Writes:} First, consider the case where transactions
do not internally abort. On commit, for each written data item $d$, we
send the last written value of $d$ ($d_v$) with the transaction ID and
$L_v$, a list of all data items and values written in the transaction,
to available replicas for $d$. The replicas immediately reply with a
response to \textit{commit}, place $d_v$ it in their respective
$d_{pending}$, and send a receipt to all replicas for items in
$L_v$. Once a server receives receipts from all replicas for all items
in $L_v$, the server sets $d_{good} = d_v$ if $d_v$'s ID is greater
than $d_{good}$'s ID. Accordingly, a new value will only appear in
$d_{good}$ when the other values written with $d_{good}$ ($L_v$) are
resident on every respective replica (though possibly in their
$d_{pending}$).\vspace{.5em}

\noindent\textit{Reads:} Each client maintains a vector of transaction
IDs $V$, with one entry per data item ($V(d)$). When a client reads a
new data item $d_i$, the server returns both $d_i$ and $L_v$, and, for
all data items $k \in L_v$, the client sets $V(k)$ to $d_i$'s ID. When
a client sends a read request for item $x$, it also sends $V(x)$, and
the replica returns \textit{either} $x_{good}$ if it matches $V(x)$ or
occurs after $V(x)$ in the transaction order or, if $x_{good}$ is
unsatisfactory, $V(x)$ from $x_{pending}$. If $V(x)$ is empty, the
server returns $x_{good}$.\vspace{.5em}

\noindent\textit{Aborts:} If we allow internal aborts, writes require
2 RTTs: one to check constraints and one to execute the above. Write
propagation and update visibility are achieved asynchronously, as in
the no-abort case.\vspace{.5em}

This implementation is entirely masterless and both reads and writes
do not block for coordination. One key property is that we do not, in general,
dictate when writes become visible (see: session guarantees,
\S\ref{sec:impossible}). Accordingly, it is highly available.\vspace{.5em}

\subsubsection{Session Guarantees}

So far, our models have not considered guarantees across transactions,
but, in the distributed systems context, many useful \textit{safety}
guarantees span multiple operations. There is no guarantee of forward
progression between transactions or any guarantees on the ordering
between transactions (other than that there exists some arbitrary
ordering).

\textit{Session guarantees} are used to describe continuity between
groups of transactions. Informally, a \textit{session} is a form of
context that consists of logical state that should persist between
operations. Users accessing a web site may want to experience a notion
of ``forward progress'' for the duration that they are logged in, and
the duration of the ``log in'' to ``log out'' period forms a
session. We can define session guarantees at several granularities
(e.g., per-transaction, per-client, per-hour) but we choose to operate
under a client-centric model.

There are several guarantees that we can make with high availability:

\vspace{.25em}\noindent\textit{\textbf{Monotonic reads}} stipulates
that a client's subsequent reads to a given object ``never return any
previous values''; reads from each item progress forward in a total
order. Under Read Uncommitted, the ordering of reads should respect the
total ordering on transactions.

\vspace{.25em}\noindent\textit{\textbf{Monotonic writes}} requires
that transactions from each individual client be serialized. Under
Read Uncommitted, the transaction order should also respect the order
in which each client submitted transactions.

\vspace{.25em}\noindent\textit{\textbf{Writes Follow Reads}} requires
that, if a client reads a write originating from transaction $T_1$
then performs transaction $T_2$, then a client can only read $T_2$'s
writes if it can also read $T_1$'s prior write (or overwritten values
for $T_1$'s write). This requirement forms the basis for potential
causality: writes follow reads obeys the ``reads-from'' relation that
captures all events that influenced each
transaction~\cite{causalmemory}. Under Read Uncommitted, the
transaction order should respect the reads-from order (note that the
total order on transactions precludes reads-from cycles).

We can achieve the above guarantees by forcing servers to wait to
reveal new writes until their respect dependencies are fulfilled on
all replicas. In the case of Monotonic Reads, servers should only
reveal new writes to a data item when all other servers responsible
for the item have also seen them. Otherwise, a client might read a
newer version, subsequently become partitioned from its prior server,
and contact a comparatively out-of-date server. In the case of
Monotonic Writes, each new write should become visible only when each
client's prior write is visible everywhere, along with Monotonic Reads
guarantees. For Writes Follow Reads, a servers should similarly wait
to reveal items until their dependencies are present
everywhere.

Effectively, this mechanism requires that clients only read from a
globally agreed upon lower bound on the writes in the system. This is
indeed highly available as a client will never block due to inability
to find a server with a sufficiently up-to-date version of a data
item. However, it does not imply that transactions will read their own
writes or provide much confidence in making forward progress through
the version history in the presence of partitions. The problem is
that, if a server is partitioned, under the highly available model, we
cannot discount the possibility that an unfortunate client will be
forced to issue her next requests against the partitioned server!

The solution to this conundrum is to give up high availability in
favor of sticky availability. Sticky availability permits two
additional models, which we first define and then prove are
unachievable in a generic highly available system:

\noindent\textit{\textbf{Read your writes}} requires
that whenever a client reads a given data item after updating it, the
read returns the updated value (or a value with a higher ID).

\vspace{.25em}\noindent\textit{\textbf{Causal consistency}} is the
combination of all of the prior session
guarantees~\cite{daudjee-session} and is also referred to as PL-2L
isolation~\cite{adya}). We can also consider arbitrary
application-defined partial orders, as in explicit
causality~\cite{explicit-socc}, if dependencies are specified at
commit time.

Read your writes is unavailable in a highly available system. Consider
a client that executes the following two transactions in succession:
\vspace{-.5em}
\begin{align*}
\small
T_1 &: w_x(1)
\\T_2 &: r_x(a)
\end{align*}
If the client executes $T_1$ against a server that is partitioned from
the rest of the other servers, the server must allow $T_1$ to
commit. If the client executes $T_2$ against the same (partitioned)
server, then it will be able to read its writes. However, if the
network topology shifts and the client can only contact a different
server which is partitioned from the rest of the cluster, then the
client will not be able to read its own writes and the system will
have to either stall indefinitely to allow the client to read her
writes (violating transactional availability) or will have to
sacrifice read your writes. Accordingly, read your writes requires
stickiness, and, because causal consistency requires read your writes,
causal consistency also requires stickiness.

Read-your-writes can be accomplished via client-side caching and
alternatively comes ``for free'' when requests are routed to a sticky
set of servers. Achieving causal consistency requires maintaining
read-your-writes along with the prior session guarantees. Both are
reasonable strategies often employed in real-world
systems~\cite{vogels-defs}, albeit at the cost of true high
availability or client-side storage.

We discuss additional uses for stickiness in Section~\ref{sec:discussion}.

\subsubsection{Additional Guarantees}

A HAT system can make \textbf{limited consistency} guarantees. It can
execute commutative and logically monotonic~\cite{needed} operations
without the risk of invalidating (also monotonic) application-level
integrity constraints. Our goal in this paper is not to sketch the
entire space of consistency models that are achievable (see
Section~\ref{sec:futurework}, however we specifically evaluate TPC-C
transaction semantics under HAT consistency guarantees in
Section~\ref{sec:Evaluation}.

As we briefly discussed in Section~\ref{sec:availabile}, a client
requiring \textbf{durability} of surviving $F$ server faults requires
at least $F+1$-availability.

Finally, to require that servers propagate writes between replicas, we
can require \textbf{convergence}, or eventual consistency for each
data item: in the absence of new mutations to a data item, in the
absence of partitions, all servers eventually agree on the value for
each item. This is typically accomplished by any number of
anti-entropy protocols, which periodically update neighboring servers
with the latest value for each data item.

\subsection{Unachievable HAT Semantics}
\label{sec:unachievable-hat}

While there are infinitely many HAT models
(Section~\ref{sec:futurework}), at this point, we have exhausted our
range of achievable, previously defined (useful) HAT semantics. Before
summarizing our possibility results, we will present impossibility
results for HATs, also defined in terms of previously identified isolation and consistency anomalies.

\subsubsection{Unachievable ACID Isolation}

* Lost Update
	* Simple proof: T1 R(x) W(x=x+1) T2 R(x) W(x=x+1); execute on opposite sides 
* Read Skew
	* Similar easy proof
* Write Skew
	* Similar easy proof
* Anti-dependency cycles
	* Serializability
* Recency guarantees on data items
	* CAP



\subsection{Summary}
\label{sec:hat-summary}

 \newcommand{\lostupdate}{$^\dagger$}
 \newcommand{\rwskew}{$^\ddagger$}
 \newcommand{\mphantom}{$^\star$}
 \newcommand{\linearizable}{$^\oplus$}

\begin{table}
\begin{tabular}{| c | p{6cm} | }\hline
HA & Transactional Atomicity (TA), Read Uncommitted (RU), Read
Committed (RC), Cut Isolation (CI), Monotonic Reads (MR), Monotonic
Writes (MW), Writes Follow Reads (WFR)\\\hline Sticky-HA & Read Your
Writes (RYW), Causal Consistency\\\hline Unavailable & Cursor
Stability (CS)\lostupdate, Read Consistency (CR)\lostupdate, Snapshot
Isolation (SI)\lostupdate, Repeatable Read (RR)\lostupdate\rwskew,
Serializability (S)\lostupdate\rwskew\mphantom, Causal
Serializability\lostupdate\rwskew\mphantom, Strict
Serializability\lostupdate\rwskew\mphantom\linearizable \\\hline
\end{tabular}
\caption{Summary of highly available, sticky highly available, and
  unavailable models considered in this paper. Unavailable models are
  labeled by cause of unavailability: preventing lost
  update\lostupdate, preventing read/write skew\rwskew, preventing
  phantoms\mphantom, and requiring linearizability\linearizable.}
\label{table:hatcompared}
\end{table}

\begin{figure}
\centering
\begin{tikzpicture}[scale=0.8]
  \tikzstyle{sticky}=[rectangle,draw=blue!50,fill=blue!20,thick]
  \tikzstyle{noha}=[ellipse,draw=red!50,fill=red!20,thick, inner sep=0pt,minimum size=12pt]

  \tikzstyle{every node}=[font=\small]

 \node[draw=none,fill=none] (ci) at (1.2, 1.6) {\underline{CI}};
 \node[draw=none,fill=none] (rc) at (1.2, .8) {\underline{RC}};
 \node[draw=none,fill=none] (ru) at (1.2, 0) {\underline{RU}};

 \node[draw=none,fill=none] (ta) at (0, 0) {\underline{TA}};
 \node[draw=none,fill=none] (mr) at (2.4, 0) {\underline{MR}};
 \node[draw=none,fill=none] (mw) at (3.6, 0) {\underline{MW}};
 \node[draw=none,fill=none] (wfr) at (4.8,0) {\underline{WFR}};
 \node at (6,0) [sticky] (ryw) {RYW};

 \node at (3.6, 2) [sticky] (causal) {Causal};
 \node[noha] (cs) at (1.7, 2.4) {CS};
 \node[noha] (rr) at (.7, 2.6) {RR};
 \node[noha] (si) at (-.4, 2.2) {SI};
 \node[noha] (cr) at (-.4, 1.2) {CR};
 \node[noha] (s) at (.5, 3.5) {S};
 \node[noha] (causal-s) at (3.6, 3.5) {Causal S};
 \node[noha] (strict-s) at (6, 3.5) {Strict S};


 \draw [->] (ru) -- (rc);
 \draw [->] (rc) -- (ci);

 \draw [->, blue] (mr) -- (causal);
 \draw [->, blue] (mw) -- (causal);
 \draw [->, blue] (wfr) -- (causal);
 \draw [->, blue] (ryw) -- (causal);

 \draw[snake=coil, segment aspect=0, segment amplitude=.75pt, segment length=2pt] (ru) -- (mr);
 \draw[snake=coil, segment aspect=0, segment amplitude=.75pt, segment length=2pt] (rc) -- (ta);
 \draw[snake=coil, segment aspect=0, segment amplitude=.75pt, segment length=2pt] (ci) -- (ta);
 \draw[snake=coil, segment aspect=0, segment amplitude=.75pt, segment length=2pt] (rc) -- (mr);
 \draw[snake=coil, segment aspect=0, segment amplitude=.75pt, segment length=2pt] (ci) -- (mr);
 \draw[snake=coil, segment aspect=0, segment amplitude=.75pt, segment length=2pt] (ta) -- (ru);
 \draw[snake=coil, blue, segment aspect=0, segment amplitude=.75pt, segment length=2pt] (ru) -- (causal);
 \draw[snake=coil, segment aspect=0, segment amplitude=.75pt, segment length=2pt] (mr) -- (mw);
 \draw[snake=coil, segment aspect=0, segment amplitude=.75pt, segment length=2pt] (wfr) -- (mw);
 \draw[snake=coil, blue, segment aspect=0, segment amplitude=.75pt, segment length=2pt] (wfr) -- (ryw);

 \draw [->, red] (ci) -- (cs);
 \draw [->, red] (cs) -- (rr);
 \draw [->, red] (ci) -- (si);
 \draw [->, red] (rr) -- (s);
 \draw [->, red] (si) -- (s);
 \draw [->, red] (ta) -- (cr);
 \draw [->, red] (rc) -- (cr);
 \draw [->, red] (cr) -- (si);
 \draw [->, red] (ta) -- (rr);
 \draw [->, red] (s) -- (causal-s);
 \draw [->, red] (causal) -- (causal-s);
 \draw [->, red] (causal-s) -- (strict-s);

\end{tikzpicture}
\label{fig:hat-order}
\caption{HAT, sticky HAT (in boxes), and unavailable models (circled)
  from Table~\protect\ref{table:hatcompared} compared graphically. Directed
  edges represent ordering by model strength, squiggle edges represent
  (transitive) compatible models, while absence of edges represents
  incomparable models. The combination of all underlined models
  represents maximum ACID HAT semantics, while the combination of
  underlined and boxed models represents maximum ACID sticky HAT
  models.}
\label{fig:hatcompared}
\end{figure}


\subsection{Discussion}
\label{sec:discussion}





TRASH

We can accomplish these three requirements in two ways. First, clients
can locally cache their reads and writes, updating caches when they
read admissible replacements from
replicas~\cite{sessionguarantees}. Second, clients can ensure
\textit{stickiness} with groups~\cite{vogels-defs} (subject to the
same groups caveat as before).  For monotonic writes, clients should
generate increasing IDs for each transaction they perform. Caching and
ID generation are both available.

To provide writes follow reads,
severs (or clients) can buffer transaction writes until the
transactions the writes depend on have been written to all
replicas~\cite{cops}. If we consider a group model, each group can
independently apply transactions in order via a (possibly sharded)
log-shipping approach~\cite{causalmemory, cops, eiger, swift}.




What about existing systems?
* Database technology developed for single-node systems; gold standard: serializability
	* Serializability is not actually highly available; give an example!
	* Much of the database literature presumes serializability and does not consider high availability.
	* For the literature that doesn't presume serializability, it's not presented in a HA context?
* Consequence: traditional systems are not optimized for high availability
	* Consider two-phase locking in a distributed environment
	* As we will see (forward reference to Evaluation), many existing transactional systems encounter similar difficulties
* Key question in this paper: What transactional semantics are highly available, and which aren't?
	* Side note: there are infinite incomparable consistency models (e.g., always return 2, always return 3, …)
	* Our goal is to unify distributed systems literature with ACID transaction model.
* Argument: Highly Available Transactional Systems (HATS) require rethinking existing models. This work is a first step.

In this sense, HATs are similar to RAID: optimizing for graceful
handling of a worst-case failure scenario improves average-case
performance.

IV. HATS and ACID (4pg)

We will show what semantics are available in HATS as well as which aren't. Our strategy will be to build up a set of properties until we reach a point of unavailability. We will start with isolation and atomicity since they are well-defined in the literature and most meaty.

A. *What's Available in Isolation and Atomicity?*

* We can prevent Dirty Write phenomena by consistently ordering transactions
	* This is basically a cross-data item convergency property, as are found in eventually consistent systems!
* Read Committed
	* Don't read dirty data by never exposing dirty data to readers!
* ANSI Repeatable Read => rename to cut isolation
	* Repeatable Read is a tricky guarantee. In ANSI SQL Spec, rather weak, but in follow-on work, it's pretty strong.
	* The property that the ANSI SQL spec talks about is about reading a "snapshot" of data items, along with your own writes.
		* A "snapshot" describes a "consistent cut" across data items; we'll call it "cut isolation"--when paired with below guarantees, makes sense.
	* This doesn't provide any constraints on writes.

* There are several properties that aren't discussed in the formal literature on isolation guarantees but which are really useful.

* Transactional atomicity: "A" in ACID
	* Once a transaction reads one of another transactions's writes, its subsequent reads will return the transactions's other writes (or a suitable "later" write).
	* Be really pedagogical here: "transactional atomicity" versus "distributed linearizability" both called "atomicity"; we actually explore the differences later, in Discussion section.
	* Talk about implementation here: it's basically uniform eager reliable delivery from distributed systems literature!
	* Note that this doesn't make recency guarantees (forward reference to later section on "impossibility")

* Session guarantees give useful guarantees within and across transactions
	* Without them, cut isolation is sort of meaningless: we can always just return NULL!
	* Several properties: monotonic writes, monotonic reads, writes follow reads, read your writes
	* We can define them within a transaction only (e.g., transactional monotonic writes) or across subsequent transactions by the same client (e.g., client monotonic writes).

* Session guarantees are tricky
	* You can implement all but read-your-writes in a R-HA system
		* Never show a write until all replicas in the system have seen it!
		* This means that, if a client switches replicas, then the replica will have a satisfying set of writes.
                * Take S3 as an example of a system where stickiness coudl have helped, but no one really talked about it!
		* Downside: when do writes become visible? Only when all replicas have seen it!
		* Side note: assuming that, if replicas can enter and leave, they can only become active once they "catch up" to global lower bound
			* We leave dynamic replica selection as Future Work (N.B. in APEs!)
	* Read-your-writes requires S-HA
		* Example: partition a client from all but one replica, have the client issue a write, which has to commit. Next, lift the partition between the client and the other replicas and partition the client and its previous replica. It can't read it's own write!
		* This also means that causality, which is MR+MW+WFR+RYW is unavailable!
		* S-HA is *doable* but it does result in substantially lowered unavailability unless one caches (e.g., bolt-on causal consistency)

\noindent\textit{Groups:} We can optimize write visibility delay by
partitioning servers and clients into groups, such that each group
contains a fully replicated set of data items and all clients within a
group contact only the servers in the group. Once all transaction
values are present within a group, values can be installed as
$d_{good}$ (as opposed to once they are present on all replicas). For
example, in a multi-datacenter setting, all clients and servers within
a datacenter may form a group. However, group-based atomicity is not
highly available according to our definition unless clients and
servers in a group fail together, as is often assumed in a
geo-replicated context~\cite{cops, eiger}.

* What do we have?
	* In Adya, we get up to Causal+Cut Isolation for S-HA, and PL-2 for R-HA.
	* In distributed systems terminology, in S-HA we have causal consistency but each transactions' updates are a cycle in the happens-before graph!
		* Stress that this is a big unification of the existing models.
	* Recap, pedagogically, why this is highly available and what this means: no locks, no central coordination, no need for RTTs; bring back to Section II.

B. *What's not available in Isolation?*

Now, at least in terms of the existing literature, we hit the limits.

Let's sketch out a few conditions, then we'll discuss them later:

Now refer to Adya's chart. No PL-SI, PL-SS, PL-2.99, etc.

C. *What's available in consistency?*

* In general, we can't maintain correctness conditions over arbitrary data items.
	* This is due to serializability problems.
	* Uniqueness, for example, is out the window

* We can evaluate locally checkable correctness criteria (e.g., check for null)
	(QUESTION: should we save this for APEs?)

* However, with semantic knowledge, we can do a little bit better
	* Commutative updates and logical monotonicity are fine; e.g., write-write conflicts don't matter
	
Example: In TPC-C New Order, the new order id assignment isn't commutative, tough. but the inventory checking *is*!

D. Durability

* If you want data to survive *F* failures, you'll have to replicate to *F* replicas.
	* This is pretty easy.

E. Summary

* Left with a bunch of binary properties: 
R-HA: Prevent Dirty Write, Read Committed, Cut Isolation, Transactional Atomicity, Transactional Monotonic Reads, Transactional Monotonic Writes, Transactional Writes-Follow-Reads, Client Monotonic Reads, Client Monotonic Writes, Client Writes-Follow-Reads
S-HA: Client Read Your Writes, Transactional Read Your Writes

* Combining all of the above results in the strongest models we've yet seen. No

*Discussion*

Stickiness and PNUTS
* As an example of a real system going from "CP" to "AP", [PNUTS dropped per-record mastering](\url{http://developer.yahoo.com/blogs/ydn/posts/2010/06/sherpa_update/#4})


* Composition of Properties can be tricky:
	* Consider ensuring RR and TA:
		* If you do this naively, you may end up returning lots of nulls
		* Give example: T1 w(x1) w(y1) T2 w(x2) w(y2) T3 r(x) r(y) => T3 better read x1, y1 or x2, y2!
	* Some are more expensive; not likely to achieve causality with less than O(clients) metadata required
		* Note that making stronger assumption about clusters, like linearizability *can* help (but then not HA!)

* Transactional Atomicity versus Linearizability
	* What's the difference between our TA and distributed systems linearizability? Is there some sort of mapping?
		* TA: writes to multiple keys are indivisible to readers
		* Linearizability: writes to single key on multiple servers are indivisible to readers
	* The differences are two-fold:
		* In linearizability (and safe and regular registers), writes are visible to clients immediately after they finish
			* HATS have no recency requirement
		* In linearizability, all clients see all writes at the same time!
			* In TA, clients see writes at different times depending on what replicas they contact
			* No such analog in traditional distributed systems literature on replica consistency
	
* S-HA Monotonic Reads and Monotonic Writes
	* In the model we discussed, you can get monotonic reads in a R-HA system by waiting until all replicas see a write.
	* This leads to poor visibility--writes take a long time to show up. Even if you didn't want to read your writes, you might want to read other peoples' writes!
	* Accordingly, there are valid reasons for S-HA session guarantees, namely, visibility.
