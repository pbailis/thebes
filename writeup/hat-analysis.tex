
\section{Highly Available Transactions}
\label{sec:hats}

HAT systems provide transactions with transactional availability and
either high availability or sticky high availability. They offer
latency and availability benefits over traditional distributed
databases, yet they cannot achieve all possible semantics. In this
section, we delineate ACID, distributed consistency, and session
consistency levels which can be achieved with high availability
(transactional atomicity, variants of Repeatable Read isolation, and
many session guarantees), those with sticky high availability (read
your writes, PRAM and causal consistency), and properties that cannot
be provided in a HAT system (those preventing Lost Update and Write
Skew, or with recency).  We present a full summary of these results in
Section~\ref{sec:hat-summary}.

As Brewer states, ``systems and database communities are separate but
overlapping (with distinct vocabulary)''~\cite{brewer-slides}. With
this challenge in mind, we build on existing properties and
definitions from the database and distributed systems literature,
providing a brief, informal explanation and example for each
guarantee. The database isolation guarantees require particular care,
since different DBMSs often use the same terminology for different
mechanisms and may provide additional guarantees in addition to our
implementation-agnostic definitions.  We draw largely on Atul Adya's
dissertation~\cite{adya} and somewhat on its predecessor work: the
ANSI SQL specification~\cite{ansi-sql} and Berenson et al.'s
subsequent critique~\cite{ansicritique}.

For brevity, we provide an informal presentation of each guarantee
here (accompanied by appropriate references) but give a full set of
formal definitions in our extended Technical Report~\cite{hat-tr}. In
our examples, we exclusively consider read and write operations,
denoting a write of value $v$ to data item $d$ as $w_d(v)$ and a read
from data item $d$ returning $v$ as $r_d(v)$. We assume that all data
items have the null value, $\bot$, at database initialization, and,
unless otherwise specified, all transactions in the examples commit.

\subsection{Achievable HAT Semantics}

To begin, we present well-known semantics that can be achieved in HAT
systems. In this section, our goal is feasibility, not performance. As
a result, we offer proof-of-concept highly available algorithms, but
our goal is \textit{not} to provide optimal or even efficient
implementations: the challenge is to provide algorithms that provide
high availability. However, we briefly study performance implications
in Section~\ref{sec:evaluation}.

\subsubsection{ACID Isolation Guarantees}

To begin, \textbf{Read Uncommitted} isolation is captured by Adya as
\textit{PL-1}. In this model, writes to each object are totally
ordered, corresponding to the order in which they are installed in the
database. In a distributed database, different replicas may receive
writes to their local copies of data at different times, but the order
in which replicas overwrite versions should be consistent with the total order for
each item. \textit{PL-1} requires that writes to different objects are
ordered consistently across transactions, prohibiting Adya's
phenomenon $G0$, also called ``Dirty Writes''~\cite{adya}. If we
create a graph of transactions with edges from one transaction to
another if the former overwrites the latter's write to the same
object, then, under Read Uncommitted, there should be no cycles in
this graph. Consider the following example:
\begin{align*}
\small\vspace{-1em}
T_1 &: w_x(1)~w_y(1)
\\T_2 &: w_x(2)~w_y(2)
\end{align*}
In this example, under Read Uncommitted, it is impossible for the
database to order $T_1$'s $w_x(1)$ before $T_2$'s $w_x(2)$ but order
$T_2$'s $w_y(2)$ before $T_1$'s $w_y(1)$. Read Uncommitted is easily
achieved via timestamping each update in a transaction with the same
logical timestamp (which is unique across transactions; e.g.,
combining a client ID and logical clock) and applying a ``last writer
wins'' conflict reconciliation policy at each replica. Later
properties will strengthen Read Uncommitted.

\textbf{Read Committed} isolation is particularly important in
practice as it is the default of many DBMSs. Centralized
implementations differ, with some based on long-duration exclusive
locks and short-duration read locks~\cite{gray-isolation} and others
based on multiple versions. The implementations often provide recency
and monotonicity properties beyond the simple meaning from the name,
which is what is expressed in the implementation-agnostic definition:
under Read Committed, transactions should not access uncommitted or
intermediate versions of data items. This prohibits both ``Dirty
Writes'', as above, and also ``Dirty Reads'' phenomena.  This
isolation is Adya's \textit{PL-2} and is formalized by prohibiting
Adya's \textit{G1\{a-c\}} (or ANSI's $P1$, or ``broad'' $P1$ [2.2]
from Berenson et al.). For instance, in the example below, $T_3$
should never see $a=1$, and, if $T_2$ aborts, $T_3$ should never read
$a=3$:
\begin{align*}
\small\vspace{-1em}
T_1 &: w_x(1)~w_x(2)
\\T_2 &: w_x(3)\\
T_3 &: r_x(a)\vspace{-1em}
\end{align*}
It is fairly easy for a HAT system to prevent ``Dirty Reads'': if each
client never writes uncommitted data to shared copies of data, then
transactions will never read each others' dirty data. As a simple
solution, clients can buffer their writes until they commit, or,
alternatively, can send them to servers, who will not deliver their
value to other readers until notified that the writes have been
committed. This implementation does not provide recency or
monotonicity guarantees but satisfies the implementation-agnostic
definition.

Several different properties have been labeled \textbf{Repeatable
  Read} Isolation. As we will show in
Section~\ref{sec:unachievable-acid}, some of these are not achievable
in a HAT system. However, the ANSI standardized
implementation-agnostic definition~\cite{ansi-sql} \textit{is}
achievable and directly captures directly the spirit of the term: if a
transaction reads the same data twice, it sees the same value each
time (preventing ``Fuzzy Read,'' or $P2$). In this paper, to
disambiguate between other definitions of ``Repeatable Read,'' we will
call this property ``cut isolation,'' since each transaction reads
from a non-changing cut, or snapshot, over the data items. If this
property holds over reads from discrete data items, we call it
\textbf{Item Cut Isolation (I-CI)}, and if we also expect a cut over
predicate-based reads (e.g., \texttt{SELECT WHERE}; preventing
Phantoms~\cite{gray-isolation}, or Berenson et al.'s
$P3/A3$~\cite{ansicritique}), we have the stronger property of
\textbf{Predicate Cut-Isolation}. In the example below, under both
levels of cut isolation, $T_3$ must read $a=1$:
\begin{align*}
\small
T_1 &: w_x(1)
\\T_2 &: w_x(2)
\\T_3 &: r_x(1)~r_x(a)
\end{align*}
It is possible to satisfy Item Cut Isolation with high availability by
having clients cache their reads such that the value that they read
never changes unless they overwrite it themselves. The cache can be
cleared at the end of each transaction and can alternatively be
accomplished on (sticky) replicas via multi-versioning. Predicate Cut
Isolation is also achievable in HAT systems via similar caching or
multi-versioning that tracks entire logical ranges of predicates
in addition to item-based reads.

\subsubsection{ACID Atomicity Guarantees}

\textbf{Transactional atomicity (TA)} is core to ACID guarantees. Although, at
least by the ACID acronym, it is not an ``isolation'' property, TA
restricts transactions' ability to view the effects of partially
completed transactions. Under TA, once some of the effects of a
transaction $T_i$ are observed by another transaction $T_j$,
thereafter, all effects of $T_i$ are observed by $T_j$. Together with
item cut isolation, TA prevents Read Skew anomalies (Berenson et al.'s
A5A~\cite{ansicritique}). As an example of TA, because $T_2$ has read
$T_1$'s write to $y$, $T_2$ must observe $b=c=1$ (or later versions
for each key):
\begin{align*}
\small
T_1 &: w_x(1)~w_y(1)~w_z(1)
\\T_2 &: r_x(a)~r_y(1)~r_x(b)~r_z(c)~\\[-1.5em]
\end{align*}
$T_2$ can also observe $a=\bot$, $a=1$, or a later version of
$a$. Notably, TA requires Read Committed isolation: observing all
effects of a transaction implicitly requires observing the final
(committed) effects of a transaction as well.

Perplexingly, discussions of TA are absent from existing treatment of
weak isolation. This is perhaps again due to the single-node context
in which prior work was developed: on a single server (or a fully
replicated database), TA is achievable via lightweight locking and/or
local concurrency control over data items~\cite{gstore}. In contrast,
in a distributed environment, TA over arbitrary groups of
non-colocated items is considerably more difficult to achieve with
high availability: servers need to ensure that, regardless of replica
selection, all of a client's accesses will read a sufficiently
up-to-date/``consistent'' set of data items. However, replicas do not
need to agree on when to choose to reveal new sets of values clients,
which would require consensus. Instead, TA only requires the
equivalent of reliable broadcast (with some additional client
metadata) and is achievable in HAT systems.

TA can be achieved via multi-versioning. If clients attach vector
clocks to every write (incrementing their own position for each
transaction) and replicas store all writes via multi-versioning, then
clients can safely determine which sets of data items to read; this
approach is adopted by Swift~\cite{swift} and, to a lesser-extent,
bolt-on causal consistency~\cite{bolton}. These systems also provide
causal consistency and are subsequently sticky available (each is
implemented via client-side caching). However, TA is achievable
without stickiness via an alternate algorithm: if servers wait to
reveal writes until they are present on all replicas, clients do not
need to be sticky.

We sketch a non-sticky TA algorithm and provide greater detail in our
technical report~\cite{hat-tr}. Each replica waits to reveal write $w$
from transaction $T$ until all of the writes from $T$ are present on
all respective replicas (are \textit{pending stable}). Once they are,
replicas asynchronously reveal their respective writes to readers. To
prevent clients from reading incomplete portions of a transaction's
writes as they are being revealed (e.g., a client reads a data item
that from a replica that has detected pending stability, then attempts
to read another item from another replica which has not yet detected
pending stability), clients include additional metadata with each
write: a list of items in the transaction along with the transaction's
timestamp (from Read Uncommitted). When a client reads a write, the
write's list of items and timestamp form a lower bound on the versions
that the client should read for other items. When a client performs
subsequent reads, it includes a timestamp with its request that
corresponds current lower bound on reads from the respective item
Replicas respond with either a write matching the timestamp or a write
with a higher timestamp that is pending stable. Clients discard any
metadata upon transaction commit, and servers keep two sets of writes
for each data item: the write with the highest timestamp that is
pending stable and a set of writes that are not yet pending stable.

This non-sticky implementation is entirely masterless and neither
reads nor writes block for coordination. Moreover, there are several
possible optimizations that, for brevity, we do not describe
here. However, one key property is that we do not, in general, dictate
when writes become visible. Accordingly, the algorithm is highly
available. We can optimize write visibility by sticking clients are
with disjoint groups of servers: once all of a transaction's writes
are pending stable within a group of servers, writes can be made
visible (as opposed to once they are pending stable with respect to
all servers). For example, in a multi-datacenter setting, all clients
and servers within a datacenter may form a sticky group.


\subsubsection{Session Guarantees}

As is typical in the database literature (with few
exceptions~\cite{daudjee-session}), our models have not yet considered
interactions \textit{across} transactions, other than that there
exists some arbitrary ordering on them (via Read Uncommitted). In the
distributed systems literature, many useful \textit{safety} guarantees
span multiple, non-transactional operations. In particular,
\textit{session guarantees} are used to describe guarantees across
transactions within a given \textit{session}, ``an abstraction for the
sequence of...operations performed during the execution of an
application''~\cite{sessionguarantees}. Informally a session describes
a context that should persist between transactions: for example, on a
social networking site, all of a user's transactions submitted between
``log in'' and ``log out'' operations might form a session.

Several session guarantees can be made with high availability:

\vspace{.5em}\noindent\textbf{{Monotonic reads}} requires that, within
a session, subsequent reads to a given object ``never return any
previous values''; reads from each item progress according to a total
order. The ordering of versions read should respect any ordering on
transactions.

\vspace{.5em}\noindent\textbf{{Monotonic writes}} requires that each
session's writes become visible in the order they were submitted by
the client. Any order on transactions should also be consistent with
any precedence that a global observer would see.

\vspace{.5em}\noindent\textbf{{Writes Follow Reads}} requires that, if
a session observes an effect of transaction $T_1$ and subsequently
commits transaction $T_2$, then another session can only observe
effects of $T_2$ if it can also observe $T_1$'s effects (or later
values that supersede $T_1$'s).  Any order on transactions should
respect the reads-from order.\vspace{.5em}

The above guarantees can be achieved by forcing servers to wait to
reveal new writes until each write's respective dependencies are
fulfilled on all replicas. This mechanism effectively ensures that all
clients read from a globally agreed upon lower bound on the versions
written. This \textit{is} highly available as a client
will never block due to inability to find a server with a sufficiently
up-to-date version of a data item. However, it does not imply that
transactions will read their own writes or, in the presence of
partitions, make forward progress through the version history. The
problem is that, if a server becomes partitioned, under the highly
available model, we must handle the possibility that an unfortunate
client will be forced to issue her next requests against the
partitioned server.

The solution to this conundrum is to give up high availability in
favor of sticky availability. Sticky availability permits three
additional guarantees, which we first define and then prove are
unachievable in a generic highly available system:

\vspace{.5em}\noindent\textbf{{Read your writes}} requires
that whenever a client reads a given data item after updating it, the
read returns the updated value (or a value with a higher ID).

\vspace{.5em}\noindent\textbf{{PRAM}} (Pipelined Random Access
Memory) provides the illusion of serializing each session's operations
and is the combination of monotonic reads, monotonic writes, and read
your writes~\cite{herlihy-art}.

\vspace{.5em}\noindent\textbf{{Causal
    consistency}}~\cite{causalmemory} is the combination of all of the
session guarantees~\cite{sessiontocausal} (alternatively, PRAM with
writes-follow-reads) and is also referred to by Adya as PL-2L
isolation~\cite{adya}).\vspace{.5em}


Read your writes is not achievable in a highly available
system. Consider a client that executes the following two transactions:
\begin{align*}
\small\vspace{-1em}
T_1 &: w_x(1)
\\T_2 &: r_x(a)\vspace{-1em}
\end{align*}
If the client executes $T_1$ against a server that is partitioned from
the rest of the other servers, the server must allow $T_1$ to
commit. If the client subsequently executes $T_2$ against the same
(partitioned) server, then it will be able to read its
writes. However, if the network topology shifts and the client can
only contact a different server that is partitioned from the server
that executed $T_1$, then the client will be unable to read its own
writes and the system will have to either stall indefinitely to allow
the client to read her writes (violating transactional availability)
or will have to sacrifice read your writes guarantees. However, if the
client remains sticky with the server that executed $T_1$, then we can
disallow this scenario. Accordingly, read your writes, and, by proxy,
causal consistency and PRAM require stickiness. Read your writes is
provided by default in a sticky system. Causality and PRAM guarantees
can be accomplished with well-known variants~\cite{causalmemory,
  bolton, cops, sessionguarantees, swift} of the prior session
guarantee algorithms we presented earlier.

\subsubsection{Additional HAT Guarantees}

In this section, we briefly discuss additional noteworthy guarantees
achievable by HAT systems.

\vspace{0.5em}
\noindent{\textbf{Consistency}} A HAT system can make limited
application-level consistency guarantees. It can often execute
commutative and logically monotonic~\cite{calm} operations without the
risk of invalidating application-level integrity constraints. We do
not attempt to sketch the entire space of \textit{application-level}
consistency properties that are achievable in HAT systems (see
Section~\ref{sec:relatedwork}) but we specifically evaluate TPC-C
transaction semantics under HAT consistency guarantees in
Section~\ref{sec:evaluation}.

%\vspace{.5em}\noindent{\textbf{Durability}} A client requiring that its
%transactions' effects survive $F$ server faults requires at least
%$F+1$ non-failing replicas; this affects availability.

\vspace{.5em}\noindent{\textbf{Convergence}} To require that the
system propagates writes between replicas, we can require convergence,
or eventual consistency for each data item: in the absence of new
mutations to a data item, in the absence of partitions, all servers
should eventually agree on the value for each item. This is typically
accomplished by any number of anti-entropy protocols, which
periodically update neighboring servers with the latest value for each
data item~\cite{antientropy}. Establishing a final value is related to
determining a total order on transaction updates, as in Read
Uncommitted.

\subsection{Unachievable HAT Semantics}
\label{sec:unachievable-hat}

While there are infinitely many HAT models
(Section~\ref{sec:relatedwork}), at this point, we have largely
exhausted the range of achievable, previously defined (and useful)
semantics that are available to HAT systems. Before summarizing our
possibility results, we will present impossibility results for HATs,
also defined in terms of previously identified isolation and
consistency anomalies. Most notably, it is impossible to
prevent Lost Update or Write Skew in a HAT system.

\subsubsection{Unachievable ACID Isolation}
\label{sec:unachievable-acid}

In this section, we demonstrate that preventing Lost Update and Write
Skew---and therefore providing Snapshot Isolation, Repeatable Read,
and one-copy serializability---inherently requires unavailability.

Berenson et al. define \textit{Lost Update} as when one
transaction $T1$ reads a given data item, a second transaction $T2$
updates the same data item, then $T1$ modifies the data item based on
its original read of the data item, ``missing'' or ``losing'' $T2$'s
newer update. Consider a database containing only the following
transactions:
\begin{align*}
\small\vspace{-1em}
T_1 &: r_x(a)~w_x(a+2)
\\T_2 &: w_x(2)\vspace{-1em}
\end{align*}
If $T_1$ reads $a=1$ but $T_2$'s write to $x$ precedes $T_1$'s write
operation, then the database will end up with $a=3$, a state that
could not have resulted in a serial execution due to $T_2$'s
``Lost Update.''

It is impossible to prevent Lost Update in a highly available
environment. Consider two clients who submit the following $T_1$ and
$T_2$ on opposite sides of a network partition:
\begin{align*}
\small\vspace{-1em}
T_1 &: r_x(100)~w_x(100+20=120)
\\T_2 &: r_x(100)~w_x(100+30=130)\vspace{-1em}
\end{align*}
Regardless of whether $x=120$ or $x=130$ is chosen by a replica, the
database state could not have arisen serial execution of $T_1$ and
$T_2$.\footnote{In this example, we assume that, as is standard in
  modern databases, replicas accept values as they are written (i.e.,
  register semantics). This particular example could be made
  serializable via the use of commutative updates
  (Section~\ref{sec:evaluation}) but the problem persists in the
  general case.}  To prevent this from happening, either $T_1$ or
$T_2$ should not have committed. Each client's respective server might
try to detect that another write occurred, but this requires knowing
the version of the latest write to $x$. In our example, this reduces
to a requirement for linearizability, which is, via Gilbert and
Lynch's proof of the CAP Theorem, provably unachievable with high
availability~\cite{gilbert-cap}.

\textbf{Write Skew} is a generalization of Lost Update to multiple
keys. It occurs when one transaction $T1$ reads a given data item $x$,
a second transaction $T2$ reads a different data item $y$, then $T1$
writes to $y$ and commits and $T2$ writes to $x$ and commits. As an
example of Write Skew, consider the following two transactions:
\begin{align*}
\small
T_1 &: r_y(0)~w_x(1)
\\T_2 &: r_x(0)~w_y(1)
\end{align*}
As Berenson et al. describe, if there was an integrity constraint
between $x$ and $y$ such that only one of $x$ or $y$ should have value
$1$ at any given time, then this write skew would violate the constraint (which is preserved in serializable executions). Write skew is a somewhat
esoteric anomaly---for example, it does not appear in
TPC-C~\cite{snapshot-serializable}---but, as a generalization of Lost
Update, it is also unavailable to HAT systems.

Their need to prevent Lost Update means that Consistent Read, Snapshot
Isolation, and Cursor Stability guarantees are all unavailable.
Repeatable Read (defined by Gray~\cite{gray-isolation}, Berenson et
al.~\cite{ansicritique}, and Adya~\cite{adya}) and One-Copy
Serializability need to prevent both Lost Update and Write Skew. These
prevention requirements mean that these guarantees are inherently
unavailable.

\subsubsection{Unavailable Recency Guarantees}

Distributed data storage systems often make various recency guarantees
on reads of data items. As we have discussed, one of the most famous
is linearizability~\cite{herlihy-art}, which states that reads will
return the last completed write to a data item, and there are several
other (weaker) variants such as safe and regular register
semantics. When applied to transactional semantics, the combination of
one-copy serializability and linearizability is called \textit{strong
  (or strict) one-copy serializability}~\cite{adya} (e.g.,
Spanner~\cite{spanner}). It is also common, particularly in systems
that allow reading from masters and slaves, to provide a guarantee
such as ``read a version that is no more than five seconds out of
date'' or similar. Unfortunately, an indefinitely long partition can
force an available system to violate any recency bound, so recency
bounds are not enforceable by HAT systems.

\subsection{Summary}
\label{sec:hat-summary}

As we summarize in Table~\ref{table:hatcompared}, a wide range of
isolation levels are achievable in HAT systems, including
transactional atomicity, cut isolation, and several session
guarantees. With sticky availability, a system can achieve read your
writes guarantees and PRAM and causal consistency. However, many other
prominent semantics, such as Snapshot Isolation, One-Copy
Serializability, and Strict Serializability cannot be achieved due to
the inability to prevent Lost Update and Write Skew phenomena.

We illustrate the hierarchy of available, sticky available, and
unavailable consistency models we have discussed in
Figure~\ref{fig:hatcompared}. Many models are simultaneously
achievable, but we find several particularly compelling. If we combine
all sticky-HAT guarantees, we have transactional, causal snapshot
reads (i.e., Causal Transactional Predicate Cut Isolation). If we
combine TA and P-CI, we have transactional snapshot reads. We can
achieve RC, MR, and RYW by simply sticking clients to servers. We can
also combine unavailable models---for example, an unavailable system
might provide PRAM and One-Copy
Serializability~\cite{daudjee-session}.

To the best of our knowledge, this is the first unification of
database ACID transactions, distributed consistency, and session
guarantee models. Interestingly, strong one-copy serializability
subsumes all other models, while considering the (large) power set of
all compatible models (e.g., the diagram depicts 96 possible HAT
combinations) hints at the vast expanse of consistency models found in
the literature. This taxonomy is not exhaustive, but we believe it
lends substantial clarity into the relationships between a large
subset of the prominent ACID and distributed consistency
models. Additional read/write transaction semantics that we have
omitted should be easily classifiable based on the available
primitives and HAT-incompatible anomaly prevention we have already
discussed.

 \newcommand{\lostupdate}{$^\dagger$}
 \newcommand{\rwskew}{$^\ddagger$}
 \newcommand{\linearizable}{$^\oplus$}

\begin{table}[t!]
\begin{tabular}{| c | p{6cm} | }\hline
HA & Transactional Atomicity (TA), Read Uncommitted (RU), Read
Committed (RC), Item Cut Isolation (I-CI), Predicate Cut Isolation
(P-CI), Monotonic Reads (MR), Monotonic Writes (MW), Writes Follow
Reads (WFR)\\\hline Sticky-HA & Read Your Writes (RYW), PRAM,
Causal\\\hline Unavailable & Cursor Stability (CS)\lostupdate,
Snapshot Isolation (SI)\lostupdate, Repeatable Read
(RR)\lostupdate\rwskew, One-Copy Serializability
(1SR)\lostupdate\rwskew, Recency\linearizable, Safe\linearizable,
Regular\linearizable, Linearizability\linearizable, Strong
1SR\lostupdate\rwskew\linearizable \\\hline
\end{tabular}
\caption{Summary of highly available, sticky highly available, and
  unavailable models considered in this paper. Unavailable models are
  labeled by cause of unavailability: preventing lost
  update\lostupdate, preventing write skew\rwskew, and requiring
  recency guarantees\linearizable.}
\label{table:hatcompared}
\end{table}

\begin{figure}[t!]
\centering
\begin{tikzpicture}[scale=0.8]
  \tikzstyle{sticky}=[rectangle,draw=blue!50,fill=blue!20,thick]
  \tikzstyle{noha}=[ellipse,draw=red!50,fill=red!20,thick, inner sep=0pt,minimum size=12pt]

  \tikzstyle{every node}=[font=\small]

 \node[draw=none,fill=none] (ici) at (1.2, 0) {I-CI};
 \node[draw=none,fill=none] (pci) at (1.65, 1.2) {P-CI};
 \node[draw=none,fill=none] (rc) at (-1.2, .8) {RC};
 \node[draw=none,fill=none] (ru) at (-1.2, 0) {RU};

 \node[draw=none,fill=none] (ta) at (-.2, 1.3) {TA};

 \node[draw=none,fill=none] (mr) at (3.6, 0) {MR};
 \node[draw=none,fill=none] (mw) at (4.8, 0) {MW};
 \node[draw=none,fill=none] (wfr) at (2.4,0) {WFR};
 \node at (6.1,0) [sticky] (ryw) {RYW};

 \node[noha](recency) at (7.7, 0) {recency};
 \node[noha](safe) at (7.7, 1) {safe};
 \node[noha](regular) at (7.7, 2) {regular};
 \node[noha](linearizable) at (7.7, 3) {linearizable};
 \node at (4.8, 2) [sticky] (causal) {causal};
 \node at (4.8, 1) [sticky] (pram) {PRAM};
 \node[noha] (cs) at (-1.2, 1.8) {CS};
 \node[noha] (rr) at (-.2, 2.7) {RR};
 \node[noha] (si) at (2.2, 2.4) {SI};
 \node[noha] (1sr) at (1.2, 3.2) {1SR};
 \node[noha] (ssr) at (3.85, 3.6) {Strong-1SR};

 \draw [->, red] (recency) -- (safe);
 \draw [->, red] (safe) -- (regular);
 \draw [->, red] (regular) -- (linearizable);
 \draw [->, red] (linearizable) -- (ssr);
 \draw [->, red] (1sr) -- (ssr);
 
 \draw [->] (ru) -- (rc);
 \draw [->] (rc) -- (ta);
 \draw [->] (ici) -- (pci);

 \draw [->, blue] (mr) -- (pram);
 \draw [->, blue] (mw) -- (pram);
 \draw [->, blue] (wfr) -- (causal);
 \draw [->, blue] (ryw) -- (pram);
 \draw [->, blue] (pram) -- (causal);

 %\draw[snake=coil, segment aspect=0, segment amplitude=.75pt, segment length=2pt] (ru) -- (mr);
 %\draw[snake=coil, segment aspect=0, segment amplitude=.75pt, segment length=2pt] (rc) -- (ta);
 %\draw[snake=coil, segment aspect=0, segment amplitude=.75pt, segment length=2pt] (ici) -- (ta);
 %\draw[snake=coil, segment aspect=0, segment amplitude=.75pt, segment length=2pt] (pci) -- (ta);
 %\draw[snake=coil, segment aspect=0, segment amplitude=.75pt, segment length=2pt] (rc) -- (mr);
 %\draw[snake=coil, segment aspect=0, segment amplitude=.75pt, segment length=2pt] (pci) -- (mr);
 %\draw[snake=coil, segment aspect=0, segment amplitude=.75pt, segment length=2pt] (ici) -- (mr);
 %\draw[snake=coil, segment aspect=0, segment amplitude=.75pt, segment length=2pt] (ta) -- (ru);
 %\draw[snake=coil, blue, segment aspect=0, segment amplitude=.75pt, segment length=2pt] (ru) -- (causal);
 %\draw[snake=coil, segment aspect=0, segment amplitude=.75pt, segment length=2pt] (mr) -- (mw);
 %\draw[snake=coil, segment aspect=0, segment amplitude=.75pt, segment length=2pt] (wfr) -- (mw);
 %\draw[snake=coil, blue, segment aspect=0, segment amplitude=.75pt, segment length=2pt] (wfr) -- (ryw);

 \draw [->, red] (rc) -- (cs);
 \draw [->, red] (cs) -- (rr);
 \draw [->, red] (pci) -- (si);
 \draw [->, red] (ici) -- (rr);
 \draw [->, red] (rr) -- (1sr);
 \draw [->, red] (si) -- (1sr);
 \draw [->, red] (ta) -- (si);
 \draw [->, red] (ta) -- (rr);
 \draw [->, red] (causal) -- (linearizable);
 \draw [->, red] (ryw) -- (safe);

\end{tikzpicture}
\label{fig:hat-order}
\caption{Partial ordering of HAT, sticky HAT (in boxes, blue), and
  unavailable models (circled, red) from
  Table~\protect\ref{table:hatcompared}. Directed edges represent
  ordering by model strength. Incomparable models can be
  simultaneously achieved, and the availability of a combination of
  models has the availability of the least available individual
  model.}\vspace{-1em}
\label{fig:hatcompared}
\end{figure}


\subsection{Additional Discussion}
\label{sec:discussion}

In this section, we discuss several subtleties in our results,
specifically addressing model composition, transactional atomicity
versus linearizability, and benefits of stickiness.

\vspace{.5em}\noindent\textbf{Model Composition} Choosing between
combinations of compatible guarantees requires care. Consider the
following transactions:
\begin{align*}
\small\vspace{-1em}
T_1 &: w_x(1)~w_y(1)
\\T_2 &: w_x(2)~w_y(2)
\\T_3 &: r_x(a)~r_y(b)\vspace{-1em}
\end{align*}
If we want to guarantee both cut isolation and transactional atomicity
and the system only executes $T_1$, $T_2$, and $T_3$, then $T_3$ needs
to read $a=b=\bot$, $a=b=1$, or $a=b=2$. This means that either the
implementation should frequently return $\bot$ (definitely undesirable
and possibly non-convergent), keep multiple versions of each data item
(necessitating potentially complicated distributed garbage
collection), or use pre-declared read sets to fetch a consistent cut
of keys before each transaction begins to execute (e.g.,
COPS~\cite{cops}). Using client-side caching can alleviate some of
these challenges~\cite{bolton, swift}, but then the system becomes
sticky high available.

Composition cost may also vary by combination. For instance, Charron-Bost
 proved that, to capture causality between $N$ communicating
processes, standard vector-based approaches face an upper bound of
$O(N)$ storage per write~\cite{charron-bost}. This means that, with
$100K$ clients, each write might be accompanied by $100K$ timestamps
per vector. This is difficult to scale. By compromising on
availability (e.g., treating a datacenter as a linearizable cluster),
this overhead can be reduced~\cite{eiger}, but it is much
cheaper to provide, say, read your writes, than full causal
consistency.

\vspace{.5em}\noindent\textbf{Linearizability and Transactional
  Atomicity} The relationship between linearizability and
transactional atomicity is non-obvious. TA dictates that writes to
multiple keys across multiple servers are made visible to readers all
at once, while linearizability dictates that writes to a single key on
multiple servers are made visible to all readers at once---what is
different? First, in linearizable (and safe and regular) systems,
writes are made visible to clients \textit{immediately} after they
finish. With transactional atomicity, there is no recency
guarantee. Second, in linearizable systems, all clients see all writes
at the same time. With TA as defined here, clients may see writes at
different times depending on which replicas they contact. We are not
aware of an analogous model in the distributed systems
literature. Accordingly, despite apparent similarities, TA is
incomparable with and much cheaper (by availability standards) than
linearizability.

\vspace{.5em}\noindent\textbf{Visibility and Stickiness} Sticky
availability can improve write \textit{visibility}: clients will be
able to safely read writes more quickly in a sticky available
system. In the model we discussed, it is possible to achieve several
properties like monotonic reads in a highly available system by
waiting to reveal a write until all servers have seen it and its
relevant dependencies. However, this incurs severe visibility
penalties---new writes will not become visible to clients in the
presence of partitions. A client that does not want to guarantee
read-your-writes (due to the sticky availability requirement) may
still wish to read other clients' writes with timeliness.
