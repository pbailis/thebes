
\section{Conclusions and Future Work}
\label{sec:conclusion}

The current state of database software offers uncomfortable and
unnecessary choices between availability and transactional semantics.
Through our analysis and experiments, we have demonstrated how goals
of high availability will remain a critical aspect of many future data
storage systems. Accordingly, we expose a broad design space of Highly
Available Transactions (HATs), which can offer the key benefits of
highly available distributed systems---``always on'' operation during
partitions and low-latency operations---while also providing a family
of transactional isolation levels and replica semantics that have been
adopted in practice.  In taxonomizing this design space, we show that
many transactional guarantees are achievable with high availability,
while a few, like preventing Lost Update and Write Skew, are not. This
offers a useful connection between storage semantics studied in the
database and distributed systems communities.

HATs raise several challenges for future study. In this paper, we have
highlighted standardized models, but we would like a more concise
description of which models are and are not achievable in HAT
systems. Similarly, while we have studied the analytical and
experiment behavior of several HAT models, there is substantial work
in further understanding the performance and design of systems within
the large set of HAT models. We also believe it is time to revisit
techniques such as Sagas~\cite{sagas} and other long-running
transaction support in order to aid HAT programming. Weakened failure
assumptions as in escrow or in the form of bounded network asynchrony
could enable richer HAT semantics at the cost of general-purpose
availability. Alternatively, there is a range of possible hybrid
systems providing strong semantics during partition-free periods and
weakened semantics during partitions. Based on our understanding of
what is desired in the field and newfound knowledge of what is
possible to achieve, we believe HATs provide a large and useful design
space to explore.

%% In this paper we expose a broad design space of Highly Available Transactional 
%% Systems (HATs), which can offer the key benefits of
%% highly available distributed systems---partition tolerance and low-latency 
%% writes---while also providing a family of transactional isolation 
%% levels and replica semantics that have been widely used in practice. 
%% In taxonomizing this design space, we show that many transactional 
%% guarantees are achievable with high availability, while a few, 
%% like preventing Lost Update and Write Skew, are not. 

%% There has been some debate in the database community regarding the importance
%% of availability in modern databases; there is also concern about the utility of
%% relaxed isolation.  Our analysis and experiments in Sections~\ref{sec:motivation} 
%% and~\ref{sec:evaluation} convince us that the availability goals of the NoSQL
%% movement will remain a critical aspect of many important systems going forward, and 
%% that the isolation levels available via HATs can have significant practical impact
%% in delivering useful transactional semantics to application developers.

%% Much of the prior research in distributed transactions has focused on prototyping 
%% individual points in the design space, with a focus on providing serializability
%% at the expense of availability.  Our goal in this paper has been to take a broader view. 
%% We taxonomize the joint design space of replica consistency and transactional isolation,
%% and identify what is possible and impossible to achieve semantically via HATs.  From
%% a research perspective, this offers a long-overdue connection between storage 
%% semantics studied in the database and distributed systems communities.
%% With this understanding in hand, the next challenge is to return to issues of 
%% performance and systems architecture within the HATs design space.  Based on our 
%% understanding of what is desired in the field and what is possible to achieve, we 
%% believe this is an area of great opportunity in the coming years.
