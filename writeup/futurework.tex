
\section{Future Work}
\label{sec:futurework}

There are several promising avenues for future work in HATs:

 A recent report from UT Austin shows that no model stronger than
 causal consistency can be achieved in a sticky highly available,
 one-way convergent system~\cite{cac}, but there are infinite highly
 available and sticky highly available models to consider. For
 example, always returning the value $1$ in response to reads is
 incomparable with (i.e., neither stronger nor weaker than) causal
 consistency: returning $1$ does not respect causality, and causal
 consistency returns values other than $1$. There are actually
 infintely many incomparable models (vacuously: return $1$, return
 $2$, and so on). \textbf{What are the limits of HAT semantics?}  Are
 there \textit{meaningful} properties that delineate the boundaries
 between availability levels?  Is there a ``meta-property'' that
 exactly characterizes what properties are achievable with high
 availability or sticky high availability?

\textbf{What are the costs and benefits within HAT designs?} There are
many possible combinations of HAT guarantees, and the performance and
ease of implementation of a HAT system will vary depending on which
guarantees it provides. For example, as we have discussed, the cost in
moving from a sticky high available implementation of highly available
guarantees to a truly highly available system seems expensive for some
models (e.g., metadata requirements or visibility penalties may
become problematic). Further experimentation regarding the set of
useful and achievable HAT guarantees will shed further light on this
question.

\textbf{Where do semantics-based concurrency control and weakened
  failure models lie in the HAT landscape?} The vast literature on
semantics-based concurrency control indicates when coordination can be
avoided while providing ``consistent'' operation for limited failure
scenarios. We have performed a narrow case study of TPC-C, but a
general taxonomization of techniques such as escrow
transactions~\cite{escrow}, Sagas~\cite{sagas}, and other long-running
transaction support could help address the challenge of programming in
a weakly consistent environment. HAT guarantees can also be
strengthened by considering weaker failure models such as networks
with bounded asynchrony. Additionally, while we have focused on
systems that provide a single availability mode, there are several
interesting hybrids to consider. For example, a system may provide
strong semantics in the absence of partitions, but, in the presence of
partitions, ``lose'' guarantees by having clients reconnect to an
available replica. This provides ``best effort'' guarantees but, to
provide availability, sacrifices them in the event of partitions.
