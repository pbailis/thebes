
\section{Future Work}
\label{sec:futurework}

We believe that there are several promising avenues for future work in
Highly Available Transaction systems:

\textbf{What is the space of consistency and isolation properties?} In
this work, we have taxonomized several well-defined and
well-documented consistency properties. This is useful as it
contextualizes decades of existing database and distributed systems
design and algorithms but does not provide guidance as to what
consistency models are the ``strongest''---or admit the smallest set
of executions---within each availability class. A recent report from
UT Austin shows that no model stronger than causal consistency can be
achieved in a sticky highly available, one-way convergent
system~\cite{cac}, but there are infinite highly available and sticky
highly available models to consider. For example, always returning the
value $1$ in response to reads is incomparable with (i.e., neither
stronger nor weaker than) causal consistency: returning $1$ does not
respect causality, and causal consistency returns values other than
$1$. Always returning the value $2$ is similarly incomparable, and, as
an example, we can enumerate models that return all integers and real
numbers--an uncountably infinite set of models. Given these infinite
models, are there meaningful properties that delineate the boundaries
between classes of highly available systems? Is there a
``meta-property'' that exactly characterizes what properties are
achievable with high availability or sticky high availability? This is
an area of ongoing research within the Berkeley HAT project.

\textbf{What are the costs and benefits within HAT designs?} There are
many possible combinations of HAT guarantees, and the performance and
ease of implementation of a HAT system will vary depending on which
guarantees it provides. For example, as we have discussed, the cost in
moving from a sticky high available implementation of highly available
guarantees to a truly highly available system seems expensive for some
models (e.g., metadata requirements or visibility requirements may
become problematic). Further experimentation regarding the set of
useful and achievable HAT guarantees will shed further light on this
question.

\textbf{Where do semantics-based concurrency control and weakened
  failure models lie in the HAT landscape?} The vast amount of
literature on semantics-based concurrency control provides hints as to
when coordination can be avoided while providing ``strongly
consistent'' outcomes for limited failure scenarios. We have performed
a narrow case study in this work, but a general taxonomization of
techniques such as escrow transactions~\cite{escrow} and
Sagas~\cite{sagas} could help answer the challenge of programming in a
weakly consistent environment. Additionally, HAT guarantees can be
strengthened by considering failure models with, say, bounded
asynchrony or service-level agreements from underlying
hardware. Additionally, while we have focused on systems that provide
a single availability mode, there are several interesting hybrids to
consider.
