

\section{Introduction}

Requirements for high availability, predictable low latency, and
simplicity of coordination protocols have led many recent distributed
database designers to renounce several guarantees provided by
traditional database systems. In particular, many modern systems forgo
providing ACID guarantees---which often provide single-system image
(SSI) programmability but suffer from unavailability and often
expensive coordination costs. The CAP Theorem, demonstrating the
incompatibility of ``always on'' operation with guarantees on data
item recency in the presence of unreliable networks, has led a large
number of systems to prioritize high availability and guaranteed low
latency, which are especially important in geo-replicated contexts. We
observe a dearth of systems providing general-purpose multi-operation
\textit{transactions} over multiple data items with useful semantics.

The relationship of the CAP Theorem and ACID semantics is poorly
understood. While serializability---the gold standard of SSI
transactions---is unavailable, databases have a long tradition of
providing weaker isolation and consistency guarantees. While these
weaker guarantees do not provide SSI for arbitrary general-purpose
read/write transactions, they often admit greater concurrency and
higher performance than comparable SSI algorithms. Accordingly, weak
isolation is widely employed, overwhelmingly as the default setting in
today's traditional ``ACID'' and up-and-coming ``NewSQL'' databases
and often as the maximum guarantees offered
(Section~\ref{sec:modernacid}). While the prevalence of these
configurations is encouraging, the current literature does not provide
guidance as to the operation of weakly consistent isolation guarantees
in a distributed context. Rather, we have a strong understanding of
weak isolation in the single-server context from which it
originated~\cite{adya, ansicritique, gray-isolation} as well as
several techniques for providing distributed
serializability~\cite{bernstein-concurrency} but we have little to no
insight into which ACID guarantees are compatible with Highly
Available Transactions (HATs), or can be provided with high
availability and low latency.

This work attempts to bridge the gap between ACID properties and
highly available systems. We analyze a range of existing ACID and
distributed consistency models to provide a rigorous taxonomization of
which models are achievable in a HAT system.  While several
guarantees, like serializability and Snapshot Isolation, are
unachievable due to reliance on preventing Lost Update and Write Skew
phenomena, many other ACID guarantees are HAT compliant, such as
providing Read Committed isolation, preventing Fuzzy Read phenomena,
and ensuring atomicity of transactional updates. We also attempt to
bridge the often distant worlds of distributed system consistency and
ACID database consistency by analyzing the relationship between
so-called session guarantees~\cite{sessionguarantees}, ``register''
semantics~\cite{herlihy-art}, and ACID properties. In doing so, we
introduce a hierarchy of highly available properties as, for example,
some guarantees require ``stickiness,'' or fate-sharing between
clients and database servers. We believe that this analysis is the
first that has been performed over such a broad range of ACID and
distributed consistency models.

While this model classification is useful, we also seek to understand
the implications for end-users of HAT systems. HATs provide high
availability and low latency but cannot offer all of the semantic
guarantees of unavailable systems---when is this trade-off worthwhile?
We attempt to objectively study these virtues and limitations of both
classic and HAT systems by performing an extensive survey of
practitioner accounts and research literature, performing experimental
analyses on modern cloud infrastructure, and analyzing representative
applications for their semantic requirements. While our results are
not definitive for all applications, they suggest that HATs offer a
one to three order of magnitude latency decrease compared to
traditional protocols and provide acceptable semantics for a wide
range of programs, particularly those with monotonic logic and
commutative updates. HAT systems can also provide arbitrary foreign
key constraints on multi-item updates and sometimes provide uniqeness
guarantees. However, particularly for programs with non-monotonic
logic, HATs may fall short, requiring more coordination-intensive
mechanisms. Concisely, we attempt to understand \textit{what benefits
  do Highly Available Transactions offer and what is their cost?}

In this paper, we make the following contributions:

\begin{itemize}
\item We present a model for high availability in a transactional
  environment, replica-based High Availability, Sticky High
  Availability, and Transactional High Availability.

\item We analyze the existing literature on ACID properties and
  replicated data semantics to show which properties are achievable in
  HAT systems. Achievable properties include transactional atomicity,
  variants of repeatable read isolation, and session guarantees. We
  also describe unavailable properties, like preventing Lost Update
  and Write Skew. We believe that this is the first such
  characterization in the literature.

\item We further motivate the study of HATs via the analysis of
  existing concurrency control algorithms, a case-study of running a
  transactional application with HATs, and via deployment of a HAT
  database prototype on public cloud infrastructure.
\end{itemize}

