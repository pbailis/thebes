

\section{Introduction}

Requirements for high availability, predictable low latency, and
simplicity of coordination protocols have led many recent distributed
database designers to renounce several properties provided by
traditional database systems~\cite{bigtable, pnuts, dynamo}. In
particular, many modern systems forgo providing ACID guarantees, which
often provide single-system image programmability but traditionally
suffer from unavailability and often expensive coordination
costs~\cite{davidson-survey}. In particular, the July 2000
announcement of the CAP Theorem~\cite{gilbert-cap}, which states that
a highly available system cannot provide recency guarantees in the
presence of network partitions, proved a rallying call for a
generation of scalable but semantically weak system designs. Over a
decade later, we observe a dearth of highly available systems
providing general-purpose multi-operation \textit{transactions} over
multiple data items with useful semantics.

While serializability, the gold standard of traditional ACID
transactions, is not achievable with high availability for general
read-write transactions~\cite{davidson-survey}, database systems have
a long tradition of providing weaker isolation and consistency
guarantees~\cite{adya, ansicritique, gray-isolation}. Due to their
concurrency and overall performance benefits, weak isolation models
are widely employed, overwhelmingly as the default setting in today's
traditional ``ACID'' and up-and-coming ``NewSQL'' databases and often
as the maximum guarantee offered (Section~\ref{sec:modernacid}). While
this weak isolation does not provide serializability for
general-purpose transactions, it is apparently strong enough to
deliver acceptable behavior to many application programmers and is
substantially stronger than the typical semantics of current highly
available systems. This raises the question: which of these semantics
can be provided while remaining highly available?

Unfortunately, the relationship of the CAP Theorem and ACID semantics
is poorly understood. We have a strong understanding of weak isolation
in the single-server context from which it originated~\cite{adya,
  ansicritique, gray-isolation} as well as several techniques for
providing distributed serializability~\cite{bernstein-concurrency,
  spanner, granola, daudjee-session, calvin}. However, the current
literature does not provide guidance as to the operation of weakly
consistent transactional guarantees in a highly available distributed
context. Accordingly, in this work, we consider the problem of
providing Highly Available Transactions (HATs), or semantic guarantees
regarding multiple operations operating on multiple data items that
are achievable with high availability and low latency, albeit with
some cost to achievable semantics. While our prior, preliminary
results indicated that some models are achievable~\cite{hat-hotos},
the literature lacks a deep understanding of what models are
available, which are not, and when this distinction matters.

This work attempts to bridge the gap between ACID properties and
highly available systems. We introduce a hierarchy of consistency
models and rigorously taxonomize their availability characteristics;
in doing so, we unify previous work on ACID properties~\cite{adya},
session guarantees~\cite{sessionguarantees}, and traditional
disributed ``register'' semantics~\cite{herlihy-art}. By identifying
the availability cost of a wide range of popular consistency
guarantees, this work allows developers to weigh the costs and
benefits of relying on particular guarantees in their applications. While
several models, like serializability and Snapshot Isolation, are
unachievable due to update conflict avoidance (preventing Lost Update
and Write Skew phenomena), many other ACID guarantees are HAT
compliant, such as providing Read Committed isolation, preventing
Fuzzy Read phenomena, and ensuring atomicity of transactional
updates. Several other guarantees, like reading one's writes, fall
into a new availability class---``sticky'' high availability---which
provides an intermediate ground between strict high availability and
majority availability.

This taxonomy allows us to weigh the costs and benefits of a wide
range of semantic guarantees, but which guarantees are worthwhile in
practice? We attempt to objectively study the virtues and limitations
of both classic and HAT systems by surveying practitioner accounts and
research literature, performing experimental analyses on modern cloud
infrastructure, and analyzing representative applications for their
semantic requirements. While our results are not definitive for all
applications, they suggest that HATs offer a one to three order of
magnitude latency decrease compared to traditional protocols and can
provide acceptable semantics for a wide range of programs,
particularly those with monotonic logic and commutative
updates~\cite{calm, blooml, crdt}. HAT systems can also enforce
arbitrary foreign key constraints on multi-item updates and sometimes
provide limited uniqueness guarantees. However, particularly for
programs with non-monotonic logic, HATs may fall short, requiring more
coordination-intensive, unavailable mechanisms. Concisely, we attempt
to understand \textit{what benefits do Highly Available Transactions
  offer and what is their cost?}

In this paper, we make the following contributions:
\begin{myitemize}
\item We model high availability in a transactional environment,
  including traditional and sticky high availability.

\item We taxonomize ACID and distributed consistency properties
  according to their availability characteristics.

\item We analyze existing concurrency control algorithms, perform a
  case-study of an existing transactional application, and
  briefly evaluate a HAT database prototype.
\end{myitemize}

