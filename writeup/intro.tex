
\section{Introduction}

The last decade has seen a significant shift in the design of
large-scale database backends, from the use of transactional
RDBMSs~\cite{bernstein-book, gray-isolation, gray-virtues} to the
widespread adoption of loosely-consistent distributed key-value
stores~\cite{bigtable, pnuts, dynamo}. The 2000 introduction of
Brewer's CAP Theorem, which stated that a highly available system
cannot provide strong consistency guarantees in the presence of
network partitions, helped spur this shift~\cite{brewer-slides}. As
formally proven~\cite{gilbert-cap}, the CAP Theorem concerns
linearizability: the ability to read the most recent write to a data
item that is replicated across servers~\cite{herlihy-art}. Yet the
implications of CAP---which concern a fairly narrow form of
distributed consistency---have often been conflated with the ability
to provide ACID database properties~\cite{brewer-slides, hn,
  foundation-article}; this misunderstanding has led to substantial
confusion regarding replica consistency, transactional isolation, and
high availability. The recent resurgence of transactional systems like
Spanner~\cite{spanner} suggests that programmers value these
semantics, but, unfortunately, existing transactional data stores do
not provide availability in the presence of
partitions~\cite{middleware-db, foundation-article,
  hstore,generalizedsnapshot, mdcc, krikellas-bargain, eiger, walter,
  calvin}.

Indeed, serializable transactions---the gold standard of traditional
ACID databases---are not achievable with high availability in the
presence of partitions~\cite{davidson-survey}. However, database
systems have a long tradition of providing weaker isolation and
consistency guarantees~\cite{adya, ansicritique, gray-virtues,
  gray-isolation, kemme-thesis}. Today's ACID and NewSQL databases
often employ weak isolation models due to their concurrency and
performance benefits; weak isolation is overwhelmingly the default
setting in these stores and is often the only option offered
(Section~\ref{sec:modernacid}). While weak isolation levels do not
provide serializability for general-purpose transactions, they are
apparently strong enough to deliver acceptable behavior to many
application programmers and are substantially stronger than the
semantics provided by current highly available systems. This raises a
natural question: which of these semantics can be provided while
remaining highly available?

Unfortunately, the relationship between high availability and ACID
semantics has not been well-explored. We have a strong understanding
of weak isolation in the single-server context from which it
originated~\cite{adya, ansicritique, gray-isolation} and many papers
offer techniques for providing distributed
serializability~\cite{bernstein-book, spanner, daudjee-session,
  hstore, krikellas-bargain, calvin, kemme-classification} or snapshot
isolation~\cite{daudjee-snapshot,generalizedsnapshot, kemme-snapshot,
  walter}. Additionally, the distributed computing, programming
languages, and parallel hardware literature contains many
consistency models for single operations on replicated objects~\cite{
  ceri-mechanism, chen-mechanism, pnuts, herlihy-art, eiger, cac,
  sessionguarantees}. However, the literature has largely ignored
semantic guarantees for multiple operations operating on multiple data
items in a highly available distributed environment.

Our main contributions in this paper are as follows. We fill the gap
between the plethora of previously proposed isolation and consistency
models and the goal of high availability.  We classify which among the
wide array of models are achievable with high availability, denoting
them as {\em Highly Available Transactions} (HATs). In doing so, we
demonstrate that although many implementations of HAT semantics are
not highly available, this is an artifact of the implementations
rather than an inherent property of the semantics. Our investigation
shows that, besides serializability, only Snapshot Isolation and
Repeatable Reads are not HAT-compliant, while most other isolation
levels are achievable with high availability. We also demonstrate that
many weak data consistency models from distributed systems are also
both HAT-compliant and simultaneously achievable with traditional ACID
properties.

Our investigation is based on both impossibility results and several
constructive proof-of-concept algorithms. For example, Snapshot
Isolation (SI) and Repeatable Reads (RR) are not HAT-compliant because
they require detecting conflicts between concurrent updates (as needed
for preventing Lost Updates or Write Skew phenomena), which we prove
is unavailable. However, Read Committed isolation, Transactional
Atomicity, and many of the weaker data consistency models from
database and distributed systems are achievable via novel algorithms
that rely on multi-versioning and limited client-side caching. For
several guarantees, such as causal consistency with phantom prevention
and ANSI Repeatable Read, we consider a modified form of high
availability in which clients ``stick to'' (i.e., have affinity with)
at least one server (a property which is often implicit in the
distributed systems literature~\cite{herlihy-art, eiger, cac} but
which requires explicit consideration in a client-server replicated
database context). This sticky availability is widely
employed~\cite{eiger, vogels-defs} but is less restrictive (and
therefore more easily achievable) than traditional high availability.

Our results demonstrate that a wide range of semantic guarantees can
be delivered with availability in face of partitions, but which
guarantees are worthwhile in practice? We study the virtues and
limitations of both classic and HAT systems by surveying practitioner
accounts and research literature, performing experimental analysis on
modern cloud infrastructure, and analyzing representative applications
for their semantic requirements. Our experiences with a HAT prototype
running across multiple geo-replicated datacenters indicate that HATs
offer a one to three order of magnitude latency decrease compared to
traditional distributed serializability protocols, and they can
provide acceptable semantics for a wide range of programs, especially
those with monotonic logic and commutative updates~\cite{calm, blooml,
  crdt}. HAT systems can also enforce arbitrary foreign key
constraints on multi-item updates and sometimes provide limited
uniqueness guarantees. However, particularly for programs with
non-monotonic logic, HATs can fall short, requiring
sometimes-unavailable mechanisms that are more
coordination-intensive. Overall, this paper quantifies the benefits of
Highly Available Transactions can offer as well as the necessary
semantic restrictions they place on applications.

%% In this paper, we make the following contributions:
%% \begin{myitemize}
%% \item We model high availability in a transactional environment,
%%   including traditional and sticky high availability.

%% \item We taxonomize ACID and distributed consistency properties
%%   according to their availability characteristics.

%% \item We analyze existing concurrency control algorithms, perform a
%%   case-study of an existing transactional application, and
%%   briefly evaluate a HAT database prototype.
%% \end{myitemize}

