

\section{Introduction}

Requirements for high availability, predictable low latency, and
simplicity of coordination protocols have led many recent distributed
database designs to renounce many guarantees provided by traditional
database systems. In particular, many systems forgo providing ACID
guarantees, which often provide single-system image (SSI)
programmability---at the provable expense of increased coordination
and, by proxy, unavailability. The formal introduction of the CAP
Theorem, which proves the incompatiblity of ``always on'' operation
with guarantees on data item recency in the presence of unreliable
networks has led to a large number of systems prioritizing high
availability, which brings with it guaranteed low latency, which is especially
important in geo-replicated contexts. Accordingly, we observe a dearth
of systems providing general-purpose multi-operation
\textit{transactions} over multiple data items with useful semantics.

The relationship of the CAP Theorem with ACID semantics is poorly
understood. While serializability---the gold standard of SSI
transactions---is unavailable, databases have a long tradition of
providing weaker isolation and consistency guarantees in order to
provide greater concurrency and higher performance. While these weaker
guarantees do not provide SSI for arbitrary general-purpose read/write
transactions, they are widely employed (Section~\ref{needed}),
overwhelmingly as the default setting in today's traditional ``ACID''
and up-and-coming ``NewSQL'' databases and often as the maximum
guarantees offered. While the prevalence of these weak isolation
configurations is encouraging, the current literature does not provide
guidance as to the operation of weakly consistent isolation guarantees
in a distributed context. Rather, we have a strong understanding of
weak isolation in the single-server context from which they originated
as well as several techniques for providing distributed
serializability but little to no guidance as to which ACID guarantees
are compatible with Highly Available Transactions (HATS), or can be
provided with high availability and low latency.

This work attempts to bridge this gap between ACID properties and
highly available systems. We analyze a range of existing ACID and
distributed consistency models in order to provide a formal
taxonomization of which models are achievalbe in a HAT system.  While
several guarantees, like serializability and Snapshot Isolation, are
unachievable due to reliance on preventing Lost Update and Write Skew
phenomena, many other ACID guarantees such as providing Read Committed
isolation, preventing Fuzzy Read phenomena, and ensuring atomicity of
transactional updates are achievable. We also attempt to bridge the
often distant worlds of distributed system consistency and ACID
database consistency by analyzing the relationship between so-called
session guarantees and ``register'' semantics. In performing this
taxonomization, we also introduce a hierarchy of highly available
system guarantees as, for example, some guarantees require
``stickiness,'' or fate-sharing between clients and database
servers. We believe that this analysis is the first that has been
performed over such a broad range of ACID and distributed consistency
models.

While this model classification is useful, we seek to understand the
implications of choosing to employ high availability. HATs provide
high availability and low latency but suffer from weakened semantic
guarantees. Traditional ACID properties provide strong semantic
guarantees but do not provide high availability or low latency. We
attempt to objectively study these virtues and limitations by
performing an extensive survey of practitioner accounts and research
literature, performing experimental analyses on modern cloud
infrastructure, and analyzing representative applications for their
semantic requirements. While our results are not definitive, they
suggest that HATs offer a one to three order of magnitude latency
decrease compared to traditional protocols, while providing acceptable
semantics for programs with monotonic and commutative
updates. Similarly, HAT guarantees can be used to ensure that
arbitrary foreign key constraints are satisfied but not necessarily
uniqueness constraints. For programs with non-monotonic transactional
requirements, HATs may fall short, necessitating greater coordination
strategies. Speed and availability are not without a price, but we
attempt to estimate the size of the tag. Succintly, \textit{what is
  benefits do highly available transactions offer and what is their
  cost?}

In this paper, we make the following contributions:

\begin{itemize}
\item We present a model for high availability in a transactional environment, including Replica-agnostic High Availability (R-HA) and Sticky High Availability (S-HA).
\item We analyze the existing literature on ACID properties (particularly, ANSI SQL Spec, Adya, Gray) to show which properties* with a focus on isolation* are achievable in HATS. Achievable properties include transactional atomicity, variants of repeatable read isolation, and session guarantees. We also describe unavailable properties, like preventing Lost Update and Write Skew. We believe that this is the first such characterization in the literature.
\item We further motivate the study of HATS via the analysis of existing concurrency control algorithms in a highly available environment as well as through deployment of an experimental database prototype on public cloud infrastructure.
\end{itemize}

