
\section{Introduction}

The last decade has seen a shift in the design of popular large-scale
database systems, from the use of transactional
RDBMSs~\cite{bernstein-book, gray-isolation, gray-virtues} to the
widespread adoption of loosely consistent distributed key-value
stores~\cite{bigtable, pnuts, dynamo}. Core to this shift was the 2000
introduction of Brewer's CAP Theorem, which stated that a highly
available system cannot provide ``strong'' consistency guarantees in
the presence of network partitions~\cite{brewer-slides}. As formally
proven~\cite{gilbert-cap}, the CAP Theorem pertains to a data
consistency model called linearizability, or the ability to read the
most recent write to a data item that is replicated across
servers~\cite{herlihy-art}. However, despite its narrow scope, the CAP
Theorem is often misconstrued as a broad result regarding the ability
to provide ACID database properties with high
availability~\cite{hat-hotos,brewer-slides, foundation-article}; this
misunderstanding has led to substantial confusion regarding replica
consistency, transactional isolation, and high availability. The
recent resurgence of transactional systems like Spanner~\cite{spanner}
suggests that programmers value transactional semantics, but,
unfortunately, existing transactional data stores do not provide
availability in the presence of partitions~\cite{orleans,foundation-article,
  hstore,spanner,eiger, walter,calvin}.

Indeed, serializable transactions---the gold standard of traditional
ACID databases---are not achievable with high availability in the
presence of partitions~\cite{davidson-survey}. However, database
systems have a long tradition of providing weaker isolation and
consistency guarantees~\cite{adya, ansicritique, gray-virtues,
  gray-isolation, kemme-thesis}. Today's ACID and NewSQL databases
often employ weak isolation models due to concurrency and performance
benefits; weak isolation is overwhelmingly the default setting in
these stores and is often the only option offered
(Section~\ref{sec:modernacid}). While weak isolation levels do not
provide serializability for general-purpose transactions, they are
apparently strong enough to deliver acceptable behavior to many
application programmers and are substantially stronger than the
semantics provided by current highly available systems. This raises a
natural question: which semantics can be provided with high
availability?

To date, the relationship between ACID semantics and high
availability has not been well explored. We have a strong
understanding of weak isolation in the single-server context from
which it originated~\cite{adya, ansicritique, gray-isolation} and many
papers offer techniques for providing distributed
serializability~\cite{bernstein-book, spanner, daudjee-session,
  hstore, calvin} or snapshot
isolation~\cite{daudjee-snapshot, walter}. Additionally, the distributed computing and parallel
hardware literature contains many consistency models for single
operations on replicated objects~\cite{pnuts, herlihy-art, eiger, cac,
  sessionguarantees}. However, the literature lends few clues for
providing semantic guarantees for multiple operations operating on
multiple data items in a highly available distributed environment.

Our main contributions in this paper are as follows. We relate the
many previously proposed isolation and consistency models to the goal
high availability, which guarantees a response from every data replica
in the presence of arbitrary partitions between them.  We classify
which among the wide array of models are achievable with high
availability, denoting them as {\em Highly Available Transactions}
(HATs). In doing so, we demonstrate that although many implementations
of HAT semantics are not highly available, this is an artifact of the
implementations rather than an inherent property of the semantics. Our
investigation shows that, besides serializability, Snapshot Isolation
and Repeatable Read isolation are not HAT-compliant, while most other
isolation levels are achievable with high availability. We also
demonstrate that many weak replica consistency models from distributed
systems are both HAT-compliant and simultaneously achievable with
several ACID properties.

Our investigation is based on both impossibility results and several
constructive, proof-of-concept algorithms. For example, Snapshot
Isolation and Repeatable Read isolation are not HAT-compliant because
they require detecting conflicts between concurrent updates (as needed
for preventing Lost Updates or Write Skew phenomena), which we show is
unavailable. However, Read Committed isolation, transactional
atomicity (Read Atomic isolation), and many of the weak consistency
models from database and distributed systems are achievable via
algorithms that rely on multi-versioning and limited client-side
caching. For several guarantees, such as causal consistency with
phantom prevention and ANSI Repeatable Read, we consider a modified
form of high availability in which clients ``stick to'' (i.e., have
affinity with) at least one server---a property which is often
implicit in the distributed systems literature~\cite{herlihy-art,
  eiger, cac} but which requires explicit consideration in a
client-server replicated database context. This sticky availability is
widely employed~\cite{eiger, vogels-defs} but is a less restrictive
model (and therefore more easily achievable) than traditional high
availability.

At a high level, the virtues of HATs are guaranteed responses from any
replica, low latency, and a range of semantic guarantees including
several whose usefulness is widely acepted such as Read
Committed. However, highly available systems are fundamentally unable
to prevent concurrent updates to shared data items and cannot provide
recency guarantees for reads. To understand when these virtues and
limitations are relevant in practice, we survey both practitioner
accounts and academic literature, perform experimental analysis on
modern cloud infrastructure, and analyze representative applications
for their semantic requirements. Our experiences with a HAT prototype
running across multiple geo-replicated datacenters indicate that HATs
offer a one to three order of magnitude latency decrease compared to
traditional distributed serializability protocols, and they can
provide acceptable semantics for a wide range of programs, especially
those with monotonic logic and commutative updates~\cite{calm,
  crdt}. HAT systems can also enforce arbitrary foreign key
constraints for multi-item updates and, in some cases, provide limited
uniqueness guarantees. However, particularly for programs with
non-monotonic logic or that are otherwise concurrency-sensitive, HATs
can fall short, requiring non-HA coordination.

We recognize that the large variety of ACID isolation levels and
distributed consistency models (and therefore in our taxonomization)
can be confusing and that the subtle distinctions between them may
appear to be primarily of academic concern. Accordingly, we offer the
following pragmatic takeaways:
\begin{introenumerate}
\item The default (and sometimes strongest) configurations of most
  widely-deployed database systems expose a range of anomalies that
  can compromise application-level consistency.

\item Most of these ``weak isolation'' models are achievable without
  sacrificing high availability if implemented correctly. However,
  none of these models prevents concurrent modifications.

\item In addition to providing a guaranteed response and horizontal
  scale-out, these HAT models allow for one to three order of
  magnitude lower latencies on current infrastructure.

\item For correct behavior, applications may require a combination of
  HAT and (ideally sparing use of) non-HAT isolation levels; future
  database designers should act accordingly.
\end{introenumerate}

%% In this paper, we make the following contributions:
%% \begin{myitemize}
%% \item We model high availability in a transactional environment,
%%   including traditional and sticky high availability.

%% \item We taxonomize ACID and distributed consistency properties
%%   according to their availability characteristics.

%% \item We analyze existing concurrency control algorithms, perform a
%%   case-study of an existing transactional application, and
%%   briefly evaluate a HAT database prototype.
%% \end{myitemize}

