
\section{Introduction}

The prior decade saw a large shift in the design of large-scale
database backends, from the use of transactional
RDBMSs~\cite{bernstein-concurrency, bernstein-book, gray-isolation,
  gray-virtues} to the widespread adoption of loosely-consistent
distributed key-value stores~\cite{bigtable, pnuts,
  dynamo}. Instrumental in this trend was the 2000 introduction of
Brewer's CAP Theorem, which stated that a highly available system
cannot provide strong consistency guarantees in the presence of
network partitions~\cite{brewer-slides}. However, as formally
proven~\cite{gilbert-cap}, the CAP Theorem concerns
linearizability~\cite{herlihy-art}, the ability to read the most
recent write to a data item that is replicated across servers. Yet
this result---regarding a fairly narrow form of distributed
consistency---is often conflated with ACID database
properties~\cite{brewer-slides, hn, foundation-article}, leading to
substantial confusion about the relationships between replica
consistency, transactional isolation, and high availability. However,
transactions \textit{are} useful for programmers as evidenced by the
recent resurgence of transactional data stores such as Google's
Spanner~\cite{spanner}. Unfortunately, these systems are not highly
available in the presence of partitions~\cite{walter, mdcc,
  foundation-article, krikellas-bargain, eiger}. (TODO:ADD more)

Indeed, serializable transactions, the gold standard of traditional
ACID databases, is indeed not achievable given a requirement for high
availability in the presence of
partitions~\cite{davidson-survey}. However, database systems have a
long tradition of providing weaker isolation and consistency
guarantees~\cite{adya, ansicritique, gray-isolation,
  gray-virtues} (TODO: kemme). Due to their concurrency and overall performance
benefits, weak isolation models are widely employed, overwhelmingly as
the default setting in today's traditional ``ACID'' and up-and-coming
``NewSQL'' databases and often as the maximum guarantee offered
(Section~\ref{sec:modernacid}). While weak isolation levels do not
provide serializability for general-purpose transactions, they are
apparently strong enough to deliver acceptable behavior to many
application programmers and are substantially stronger than the
semantics of current highly available systems. This raises a natural
question: which of these alaternative semantics can be provided while
remaining highly available?

Unfortunately, the relationship of the CAP Theorem and ACID semantics
has not been well explored. We have a strong understanding of weak
isolation in the single-server context from which it
originated~\cite{adya, ansicritique, gray-isolation} and many papers
offer techniques for providing distributed
serializability~\cite{bernstein-concurrency, kemme-classification,
  spanner, granola, daudjee-session, krikellas-bargain, calvin} or
snapshot isolation~\cite{daudjee-snapshot,generalizedsnapshot,
  kemme-snapshot, walter}. The distributed computing, programming
language, and parallel hardware literature also contains many
consistency models for single operations on replicated
objects~\cite{pnuts, herlihy-art, cops, eiger, cac, sessionguarantees,
  ceri-mechanism, chen-mechanism}. However, the literature has largely
ignored semantic guarantees for multiple operations operating on
multiple data items in a highly available distributed environment.

Our main contributions are as follows. We fill gap between, on one
hand, the plethora of previously proposed isolation and consistency
models, and, on the other hand, the goal of high availability.  We
classify which, among the wide array models, are highly available,
denoting them as {\em Highly Available Transactions (HATs)}. In doing
so, we demonstrate that many \textit{implementations} of HAT semantics
are not highly available but that their unavailability is not
\textit{inherently} required. Our investigation shows, somewhat
surprisingly, that besides serializability, only Snapshot Isolation
(SI) and Repeatable Reads (RR) are unavailable. We also show that many
of the weaker data consistency models from distributed systems are
HAT-compliant.

More precisely, highly available systems cannot detect conflicts
between concurrent updates (as needed for preventing Lost Updates or
Write Skew phenomena). However, Read Committed isolation,
Transactional Atomicity, and many of the weaker data consistency
models from database and distributed systems can be implemented in a
highly available manner. For several of these guarantees, such as
Transactional Causal Predicate Cut Isolation, we consider a modified
form of high availiabity in which clients ``stick to'' (i.e., have
affinity with) at least one server (a property which is often implicit
in the distributed systems literature~\cite{herlihy-art, cops, eiger,
  cac} but which requires \textit{explicit} consideration in a
client-server replicated database context). This sticky availability
is easily implementatable and widely employed~\cite{cops, eiger,
  vogels-defs} but is strictly less available than traditional
definitions of high availability.

This taxonomy allows us to determine that a wide range of semantic
guarantees can be delivered with availability in face of partitions,
but which guarantees are worthwhile in practice? We attempt to
objectively study the virtues and limitations of both classic and HAT
systems by surveying practitioner accounts and research literature,
performing experimental analyses on modern cloud infrastructure, and
analyzing representative applications for their semantic
requirements. While our results are not definitive, they suggest that
HATs offer a one to three order of magnitude latency decrease compared
to traditional distributed serializability protocols, and they can
provide acceptable semantics for a wide range of programs, especially
those with monotonic logic and commutative updates~\cite{calm, blooml,
  crdt}. HAT systems can also enforce arbitrary foreign key
constraints on multi-item updates and sometimes provide limited
uniqueness guarantees. However, particularly for programs with
non-monotonic logic, HATs may fall short, requiring
sometimes-unavailable mechanisms that are more
coordination-intensive. Overall, this paper quantifies the benefits
Highly Available Transactions can offer as well as the necessary
semantic restrictions they place on applications.

%% In this paper, we make the following contributions:
%% \begin{myitemize}
%% \item We model high availability in a transactional environment,
%%   including traditional and sticky high availability.

%% \item We taxonomize ACID and distributed consistency properties
%%   according to their availability characteristics.

%% \item We analyze existing concurrency control algorithms, perform a
%%   case-study of an existing transactional application, and
%%   briefly evaluate a HAT database prototype.
%% \end{myitemize}

