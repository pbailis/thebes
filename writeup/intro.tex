

\section{Introduction}

In July 2000, the CAP Theorem~\cite{gilbert-cap} was crisply
formulated that a highly available system cannot provide strong
consistency guarantees in the presence of network partitions. This
became a rallying call for a generation of highly scalable but
semantically weak system designs. The NoSQL movement embraced this
mantra, leading to many production quality databases that eschewed
strong consistency guarantees, but could scale-out to many
servers~\cite{bigtable, pnuts, dynamo}. The consistency concept used
in the CAP theorem was proposed by the distributed systems community,
and concerned providing strong recency guarantees of a data item that
is replicated across servers.  However, the NoSQL movement implicitly
took it to imply that transactions are also unavailable in the
presence of network partitions. Over a decade later, we observe a
dearth of highly available systems providing general-purpose
multi-operation \textit{transactions} over multiple data items with
useful semantics.


 Serializability, the gold standard of traditional ACID transactions,
 is indeed not achievable with the requirement of high availability
 the in the presence of partitions~\cite{davidson-survey}. However,
 database systems have a long tradition of providing weaker isolation
 and consistency guarantees~\cite{adya, ansicritique,
   gray-isolation}. Due to their concurrency and overall performance
 benefits, weak isolation models are widely employed, overwhelmingly
 as the default setting in today's traditional ``ACID'' and
 up-and-coming ``NewSQL'' databases and often as the maximum guarantee
 offered (Section~\ref{sec:modernacid}). While this weak isolation
 does not provide serializability for general-purpose transactions, it
 is apparently strong enough to deliver acceptable behavior to many
 application programmers and is substantially stronger than the
 typical semantics of current highly available systems. This raises
 the question: which of these semantics can be provided while always
 remaining highly available?

Unfortunately, the relationship of the CAP Theorem and ACID semantics
is poorly understood. We have a strong understanding of weak isolation
in the single-server context from which it originated~\cite{adya,
  ansicritique, gray-isolation} as well as several techniques for
providing distributed serializability~\cite{bernstein-concurrency,
  spanner, granola, daudjee-session, calvin}.  However, the current
literature does not provide guidance as to the operation of weakly
consistent transactional guarantees in a replicated context. To make
matters worse, the distributed computing literature has produced a
large body of work on weak data consistency models for single
replicated objects. Accordingly, in this work, we consider the problem
of providing understanding semantic guarantees regarding multiple
operations operating on data items that are achievable with high
availability and low latency, albeit with some cost to achievable
semantics. While our prior, preliminary results indicated that some
models are achievable~\cite{hat-hotos}, the literature lacks a deep
understanding of what models are available, which are not, and when
this distinction matters.

This work attempts to bridge the gap between, one the one hand, the
plethora of isolation and consistency models proposed in the
transactions and distributed systems literature, and, on the other hand
highly available systems. We introduce a hierarchy of consistency
models and rigorously taxonomize their availability characteristics;
in doing so, we unify previous work on ACID properties~\cite{adya},
session guarantees~\cite{sessionguarantees}, and traditional
distributed ``register'' semantics~\cite{herlihy-art}. 


Our main contributions are the following. We classify among the wide
array of isolation and consistency models those that are highly
available, denoting them as {\em Highly Available Transactions
  (HATs)}.  In doing so, we show that many implementations that
provide HAT-semantics are not available, but that the unavailability
is {\em not} inherent to their semantic guarantees. Thus, we provide
highly available implementations for HATs.  Furthermore, our
investigation shows, somewhat surprisingly, that besides
serializability, only Snapshot Isolation (SI) and Repeatable Reads
(RR) are unavailable. The reason for their unavailability stems from
their requirement to avoid update conflict (preventing Lost Updates
and Write Skew phenomena). Finally, we show that many of the weaker
data consistency models from distributed systems can be implemented in
a highly available manner as well. For that, the model of the system
has to be changed to include clients, which cannot be mobile, but
rather have to {\em stick} to one server, something which is implicit
in much of the previous work on distributed systems.


This taxonomy allows us to weigh the costs and benefits of a wide
range of semantic guarantees, but which guarantees are worthwhile in
practice? We attempt to objectively study the virtues and limitations
of both classic and HAT systems by surveying practitioner accounts and
research literature, performing experimental analyses on modern cloud
infrastructure, and analyzing representative applications for their
semantic requirements. While our results are not definitive for all
applications, they suggest that HATs offer a one to three order of
magnitude latency decrease compared to traditional protocols and can
provide acceptable semantics for a wide range of programs, especially
those with monotonic logic and commutative updates~\cite{calm, blooml,
  crdt}. HAT systems can also enforce arbitrary foreign key
constraints on multi-item updates and sometimes provide limited
uniqueness guarantees. However, particularly for programs with
non-monotonic logic, HATs may fall short, requiring more
coordination-intensive, unavailable mechanisms. Concisely, we attempt
to understand \textit{what benefits do Highly Available Transactions
  offer and what is their cost?}

%% In this paper, we make the following contributions:
%% \begin{myitemize}
%% \item We model high availability in a transactional environment,
%%   including traditional and sticky high availability.

%% \item We taxonomize ACID and distributed consistency properties
%%   according to their availability characteristics.

%% \item We analyze existing concurrency control algorithms, perform a
%%   case-study of an existing transactional application, and
%%   briefly evaluate a HAT database prototype.
%% \end{myitemize}

